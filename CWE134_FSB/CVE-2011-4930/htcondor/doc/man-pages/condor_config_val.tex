\begin{ManPage}{\label{man-condor-config-val}\Condor{config\_val}}{1}
{Query or set a given Condor configuration variable}
\Synopsis \SynProg{\Condor{config\_val}}
\oOpt{options}
\oOpt{-config}
\oOpt{-verbose}
\Arg{variable}
\oArg{variable \Dots}

\SynProg{\Condor{config\_val}}
\oOpt{options}
\OptArg{-set}{string}
\oArg{string \Dots}

\SynProg{\Condor{config\_val}}
\oOpt{options}
\OptArg{-rset}{string}
\oArg{string \Dots}

\SynProg{\Condor{config\_val}}
\oOpt{options}
\OptArg{-unset}{variable}
\oArg{variable \Dots}

\SynProg{\Condor{config\_val}}
\oOpt{options}
\OptArg{-runset}{variable}
\oArg{variable \Dots}

\SynProg{\Condor{config\_val}} \oOpt{options} \Opt{-tilde}

\SynProg{\Condor{config\_val}} \oOpt{options} \Opt{-owner}

\SynProg{\Condor{config\_val}} \oOpt{options} \Opt{-config}

\SynProg{\Condor{config\_val}} \oOpt{options} \Opt{-dump}

\index{Condor commands!condor\_config\_val}
\index{condor\_config\_val command}

\Description

\Condor{config\_val} can be used to quickly see what the current
Condor configuration is on any given machine.  Given a list of
variables, \Condor{config\_val} will report what each of these
variables is currently set to.  If a given variable is not defined,
\Condor{config\_val} will halt on that variable, and report that it is
not defined.  By default, \Condor{config\_val} looks in the local
machine's configuration files in order to evaluate the variables.

\Condor{config\_val} can also be used to quickly set configuration
variables for a specific daemon on a given machine.  Each daemon
remembers settings made by \Condor{config\_val}.  The configuration
file is not modified by this command.  Persistent settings remain when
the daemon is restarted.  Runtime settings are lost when the daemon is
restarted.  In general, modifying a host's configuration with
\Condor{config\_val}  
requires the \DCPerm{CONFIG} access level, which is disabled on all
hosts by default.
Administrators have more
fine-grained control over which access levels can modify which
settings.
See section~\ref{sec:Config-Security} on
page~\pageref{sec:Config-Security} for more details on security settings.

The \Opt{-verbose} option displays the configuration
file name and line number where a configuration variable is defined.

Any changes made by \Condor{config\_val} will not take effect
until \Condor{reconfig} is invoked.

It is generally wise to test a new configuration on a single
machine to ensure that no syntax or other errors in the
configuration have been made before the reconfiguration of many machines.  
Having bad syntax or invalid configuration settings is a fatal error
for Condor daemons, and they will exit.
It is far better to discover such a problem on a single machine than to
cause all the Condor daemons in the pool to exit.

The \Opt{-set} option sets one or more persistent configuration file entries.
The \Arg{string} must be a single argument, so enclose it in double quote marks.
A string must be of the form \Expr{"variable = value"}.
Use of the \Opt{-set} option implies the use of configuration variables
\Macro{SETTABLE\_ATTRS\Dots} (see \ref{param:SettableAttrs}),
\Macro{ENABLE\_PERSISTENT\_CONFIG} (see \ref{param:EnablePersistentConfig}),
and \Macro{HOSTALLOW\Dots} (see \ref{param:HostAllow}).

The \Opt{-rset} option sets one or more runtime configuration file entries.
The \Arg{string} must be a single argument, so enclose it in double quote marks.
A string must be of the form \Expr{"variable = value"}.
Use of the \Opt{-rset} option implies the use of configuration variables
\Macro{SETTABLE\_ATTRS\Dots} (see \ref{param:SettableAttrs}),
\Macro{ENABLE\_RUNTIME\_CONFIG} (see \ref{param:EnableRuntimeConfig}),
and \Macro{HOSTALLOW\Dots} (see \ref{param:HostAllow}).

The \Opt{-unset} option changes one or more persistent configuration file
entries to their previous value.

The \Opt{-runset} option changes one or more runtime configuration file
entries to their previous value.

The \Opt{-tilde} option displays the path to the Condor home directory.

The \Opt{-owner} option displays the owner of the \Condor{config\_val} process.

The \Opt{-config} option displays the current configuration files in use.

The \Opt{-dump} option displays a list of all of the defined macros
in the configuration files found by \Condor{config\_val}, along with
their values. If the \Opt{-verbose} option is supplied as well,
then the specific configuration file which defined each macro,
along with the line number of its definition is also printed. 
\Note The output of this argument is likely to change 
in a future revision of Condor.

\begin{Options}
  \OptItem{\OptArg{-name}{machine\_name}}{ Query the specified
    machine's \Condor{master} daemon for its configuration. 
    Does not function together with any of the options:
    \Opt{-dump}, \Opt{-config}, or \Opt{-verbose}. }
  \OptItem{\OptArg{-pool}{centralmanagerhostname[:portnumber]}}
    { Use the given central manager and an optional port number
    to find daemons. }
  \OptItem{\OptArg{-address}{\Sinful{ip:port}}}
    { Connect to the given IP address and port number. }
  \OptItem{\Opt{-master \Bar -schedd \Bar -startd \Bar -collector \Bar -negotiator}}
    {The specific daemon to query. }
  \OptItem{\Opt{-local-name}}
    {Inspect the values of attributes that use local names.}
\end{Options}

\ExitStatus

\Condor{config\_val} will exit with a status value of 0 (zero) upon success,
and it will exit with the value 1 (one) upon failure.

\Examples

Here is a set of examples to show a sequence of operations using 
\Condor{config\_val}.
To request the \Condor{schedd} daemon on host perdita
to display the value of the \MacroNI{MAX\_JOBS\_RUNNING} configuration variable:
\footnotesize
\begin{verbatim}
   % condor_config_val -name perdita -schedd MAX_JOBS_RUNNING
   500
\end{verbatim}
\normalsize

To request the \Condor{schedd} daemon on host perdita
to set the value of the \MacroNI{MAX\_JOBS\_RUNNING} configuration variable
to the value 10.
\footnotesize
\begin{verbatim}
   % condor_config_val -name perdita -schedd -set "MAX_JOBS_RUNNING = 10"
   Successfully set configuration "MAX_JOBS_RUNNING = 10" on 
   schedd perdita.cs.wisc.edu <128.105.73.32:52067>.
\end{verbatim}
\normalsize

A command that will implement the change just set in the previous
example.
\footnotesize
\begin{verbatim}
   % condor_reconfig -schedd perdita
   Sent "Reconfig" command to schedd perdita.cs.wisc.edu
\end{verbatim}
\normalsize

A re-check of the configuration variable reflects the change implemented:
\footnotesize
\begin{verbatim}
   % condor_config_val -name perdita -schedd MAX_JOBS_RUNNING
   10
\end{verbatim}
\normalsize

To set the configuration variable \MacroNI{MAX\_JOBS\_RUNNING}
back to what it was before the command to set it to 10:
\footnotesize
\begin{verbatim}
   % condor_config_val -name perdita -schedd -unset MAX_JOBS_RUNNING
   Successfully unset configuration "MAX_JOBS_RUNNING" on 
   schedd perdita.cs.wisc.edu <128.105.73.32:52067>.
\end{verbatim}
\normalsize

A command that will implement the change just set in the previous
example.
\footnotesize
\begin{verbatim}
   % condor_reconfig -schedd perdita
   Sent "Reconfig" command to schedd perdita.cs.wisc.edu
\end{verbatim}
\normalsize

A re-check of the configuration variable reflects that variable
has gone back to is value before initial set of the variable:
\footnotesize
\begin{verbatim}
   % condor_config_val -name perdita -schedd MAX_JOBS_RUNNING
   500
\end{verbatim}
\normalsize

\end{ManPage}
