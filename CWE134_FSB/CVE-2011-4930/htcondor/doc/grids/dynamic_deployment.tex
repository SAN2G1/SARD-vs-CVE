%%%%%%%%%%%%%%%%%%%%%%%%%%%%%%%%%%%%%%%%%%%%%%%%%%%%%%%%%%%%%%%%%%%%%%%%%%%%%%%%
\section{\label{sec:Dynamic-Deployment}Dynamic Deployment}
%%%%%%%%%%%%%%%%%%%%%%%%%%%%%%%%%%%%%%%%%%%%%%%%%%%%%%%%%%%%%%%%%%%%%%%%%%%%%%%%
\index{Dynamic Deployment}
\index{Deployment commands}
\index{Condor commands!condor\_cold\_start}

Dynamic deployment is a mechanism that allows rapid, automated
installation and startup of Condor resources on a given machine.  In
this way any machine can be added to a Condor pool, to provide any
role: central manager, execute site, or submit point.  The dynamic
deployment tool set also provides tools to remove a machine from the
pool without leaving any residual effects on the machine: leftover
installations, log files, or working directories.

%%%%%%%%%%%%%%%%%%%%%%%%%%%%%%%%%%%%%%%%%%%%%%%%%%%%%%%%%%%%%%%%%%%%%%%%%%%%%%%%
\subsection{What does it do?}
%%%%%%%%%%%%%%%%%%%%%%%%%%%%%%%%%%%%%%%%%%%%%%%%%%%%%%%%%%%%%%%%%%%%%%%%%%%%%%%%

The \Condor{cold\_start} program determines the operating system and
architecture of the target machine, then downloads the correct
installation package from an ftp, http, or grid ftp site.  The program
installs Condor from this package and creates a local working
directory for Condor to run in.  After this \Condor{cold\_start}
begins running Condor in a manner which allows easy and reliable
shutdown.  See page~\pageref{man-condor-cold-start} for more details.

\Condor{cold\_stop}, on the other hand, reliably shuts down and
uninstalls a previously dynamically installed Condor instance.
\Condor{cold\_stop} begins by safely and reliably shutting off the
running Condor installation.  The program ensures that Condor has
completely shut down before continuing, and optionally ensures that
there are no queued jobs at this site.  Next, \Condor{cold\_stop}
removes and optionally archives the Condor working directories,
including the \File{log} directory.  These archives can be stored to a
mounted file system or to a grid ftp site.  After which,
\Condor{cold\_stop} uninstalls the Condor executables and libraries.
The end result is that the machine resources are left unchanged after
a dynamic deployment leaves.  See page~\pageref{man-condor-cold-stop}
for more details.

%%%%%%%%%%%%%%%%%%%%%%%%%%%%%%%%%%%%%%%%%%%%%%%%%%%%%%%%%%%%%%%%%%%%%%%%%%%%%%%%
\subsection{What do I need to do?}
%%%%%%%%%%%%%%%%%%%%%%%%%%%%%%%%%%%%%%%%%%%%%%%%%%%%%%%%%%%%%%%%%%%%%%%%%%%%%%%%

Dynamic deployment is primarily designed for the Condor power user.
In other words, when design choices were being made between
functionality and ease-of-use, functionality always prevailed.

Like every instance of Condor, a dynamically deployed installation
requires a Condor configuration file.  If you wish to add the target
machine to a previously created Condor pool, the global configuration
file for that pool is a good starting point.  Any modifications to the
standard configuration can be made in a separate local configuration
file for the dynamic deployment.  The global configuration file must
be placed on an ftp, http, grid ftp, or file server where
\Condor{cold\_start} can access it.  The local configuration file
should be on a file system accessible by the target machine.  There
are some specific configuration variables that should be set for
dynamic deployment.  First, you should provide a list of executables
and directories which must be present for Condor to start on the
target machine; note this does not affect what is installed only
whether the startup is successful.  These executables and directories
are provided as comma-separated list assigned to the following
configuration variables: \Macro{DEPLOYMENT\_REQUIRED\_EXECS} and
\Macro{DEPLOYMENT\_REQUIRED\_DIRS}.  You may also optionally specify:
\Macro{DEPLOYMENT\_RECOMMENDED\_EXECS} and
\Macro{DEPLOYMENT\_RECOMMENDED\_DIRS}.  Directories and executables
specified by the \MacroNI{RECOMMENDED} configuration variables will
not cause the startup to fail if not present, but will print a warning
message.  Below is an example from a dynamic deployment of a Condor
submit node:

\footnotesize
\begin{verbatim}
DEPLOYMENT_REQUIRED_EXECS       = MASTER, SCHEDD, PREEN, STARTER, \
                                  STARTER_STANDARD, SHADOW, \
                                  SHADOW_STANDARD, GRIDMANAGER, GAHP, CONDOR_GAHP
DEPLOYMENT_REQUIRED_DIRS        = SPOOL, LOG, EXECUTE
DEPLOYMENT_RECOMMENDED_EXECS    = CREDD
DEPLOYMENT_RECOMMENDED_DIRS     = LIB, LIBEXEC
\end{verbatim}
\normalsize

In addition to which executables must be present, the user should
specify which Condor services should be started.  This is done through
the \MacroNI{DAEMON\_LIST} configuration variable.  Another excerpt
from dynamic submit node deployment configuration:

\footnotesize
\begin{verbatim}
DAEMON_LIST                     = MASTER, SCHEDD
\end{verbatim}
\normalsize

Finally, the user must detail in the configuration file the location
of the dynamically installed Condor executables.  This is tricky since
we cannot know the location until they are installed.  Therefore, the
startup process sets a configuration variable
\Macro{DEPLOYMENT\_RELEASE\_DIR}, which corresponds to the location of
the dynamic Condor installation.  If, as is often the case, the
configuration file specifies the location of Condor executables in
relation to the \MacroNI{RELEASE\_DIR} variable, the configuration can
be made dynamically deployable by setting \MacroNI{RELEASE\_DIR} to
\MacroNI{DEPLOYMENT\_RELEASE\_DIR}.  Another excerpt from dynamic
submit node deployment configuration:

\footnotesize
\begin{verbatim}
RELEASE_DIR = $(DEPLOYMENT_RELEASE_DIR)
\end{verbatim}
\normalsize

In addition to setting up the configuration, the user must also
determine where the installation package will reside.  The
installation package can be in either tar or tar'ed gzip form and may
reside on a ftp, http, grid ftp, or file server.  You must create this
installation package yourself by tar'ing up the binaries and libraries
you will need and placing them on the appropriate server.  The
binaries can be tar'ed in a flat structure or within \File{bin} and
\File{sbin}.  Below is an example structure for a dynamic deployment
of the Condor schedd.

\footnotesize
\begin{verbatim}
% tar tfz latest-i686-Linux-2.4.21-37.ELsmp.tar.gz
bin/
bin/condor_config_val
bin/condor_q
sbin/
sbin/condor_preen
sbin/condor_shadow.std
sbin/condor_starter.std
sbin/condor_schedd
sbin/condor_master
sbin/condor_gridmanager
sbin/gt4_gahp
sbin/gahp_server
sbin/condor_starter
sbin/condor_shadow
sbin/condor_c-gahp
sbin/condor_off 
\end{verbatim}
\normalsize
