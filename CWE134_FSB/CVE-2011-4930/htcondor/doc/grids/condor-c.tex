\index{Condor-C|(}

%%%%%%%%%%%%%%%%%%%%%%%%%%%%%%%%%%%%%%%%%%%%%%%%%%%%%%%%%%%%%%%%%%%%%%%%%%%
\subsection{\label{sec:Condor-C}Condor-C, The condor Grid Type }
%%%%%%%%%%%%%%%%%%%%%%%%%%%%%%%%%%%%%%%%%%%%%%%%%%%%%%%%%%%%%%%%%%%%%%%%%%%

\index{grid computing!Condor-C}
Condor-C allows jobs in one machine's job queue to
be moved to another machine's job queue.
These machines may be far removed from each other,
providing powerful grid computation mechanisms,
while requiring only Condor software and its configuration.

Condor-C is highly resistant to network disconnections and machine failures on both the submission and remote sides.
An expected usage
sets up Personal Condor on a laptop,
submits some jobs that are sent to a Condor pool,
waits until the jobs are staged on the pool,
then turns off the laptop.
When the laptop reconnects at a later time,
any results can be pulled back.

Condor-C scales gracefully when compared with Condor's flocking
mechanism.
The machine upon which jobs are submitted
maintains a single process and network connection to a remote machine,
without regard to the number
of jobs queued or running.

%%%%%%%%%%%%%%%%%%%%%%%%%%%%%%%%%%%%%%%%%%%%%%%%%%%%%%%%%%%%%%%%%%%%%%%%%%%
\subsubsection{\label{sec:Condor-C-Config}Condor-C Configuration}
%%%%%%%%%%%%%%%%%%%%%%%%%%%%%%%%%%%%%%%%%%%%%%%%%%%%%%%%%%%%%%%%%%%%%%%%%%%
\index{Condor-C!configuration}
There are two aspects to configuration to enable the
submission and execution of Condor-C jobs.
These two aspects correspond to the endpoints of the 
communication: there is the machine from which jobs are
submitted, and there is the remote machine upon which the
jobs are placed in the queue (executed).

Configuration of a machine from which jobs are submitted
requires a few extra configuration variables:

\footnotesize
\begin{verbatim}
CONDOR_GAHP=$(SBIN)/condor_c-gahp
C_GAHP_LOG=/tmp/CGAHPLog.$(USERNAME)
C_GAHP_WORKER_THREAD_LOG=/tmp/CGAHPWorkerLog.$(USERNAME)
\end{verbatim}
\normalsize

\index{Condor GAHP}
\index{GAHP (Grid ASCII Helper Protocol)}
The acronym GAHP stands for Grid ASCII Helper Protocol.
A GAHP server provides grid-related services for a
variety of underlying middle-ware systems.
The configuration variable \Macro{CONDOR\_GAHP}
gives a full path to the GAHP server utilized by Condor-C.
The configuration variable \Macro{C\_GAHP\_LOG} defines
the location of the log that the Condor GAHP server writes.
The log for the Condor GAHP is written as the user on whose
behalf it is running; thus the
\Macro{C\_GAHP\_LOG} configuration variable must point to a location the end
user can write to.

A submit machine must also have a \Condor{collector} daemon to which the
\Condor{schedd} daemon can submit a query.
The query is for the location (IP address and port)
of the intended remote machine's \Condor{schedd} daemon.
This facilitates communication between the two machines.
This \Condor{collector} does not need to be the same collector
that the local \Condor{schedd} daemon reports to.

The machine upon which jobs are executed 
must also be configured correctly.
This machine must be running a \Condor{schedd} daemon.
Unless specified explicitly in a submit file, 
\MacroNI{CONDOR\_HOST} must point to a 
\Condor{collector} daemon that it can write to,
and the machine upon which jobs are submitted can read from.
This facilitates communication between the two machines.

An important aspect of configuration is the security 
configuration relating to authentication.
Condor-C on the remote machine relies on an
authentication protocol to
know the identity of the user under which to run a job.
The following is a working example
of the security configuration for authentication.
This authentication method, CLAIMTOBE, 
trusts the identity claimed by a host or IP address.

\footnotesize
\begin{verbatim}
SEC_DEFAULT_NEGOTIATION = OPTIONAL
SEC_DEFAULT_AUTHENTICATION_METHODS = CLAIMTOBE
\end{verbatim}
\normalsize


%%%%%%%%%%%%%%%%%%%%%%%%%%%%%%%%%%%%%%%%%%%%%%%%%%%%%%%%%%%%%%%%%%%%%%%%%%%
\subsubsection{\label{sec:Condor-C-Submit}Condor-C Job Submission}
%%%%%%%%%%%%%%%%%%%%%%%%%%%%%%%%%%%%%%%%%%%%%%%%%%%%%%%%%%%%%%%%%%%%%%%%%%%
\index{Condor-C!job submission}
\index{universe!grid}
Job submission of Condor-C jobs is the same as for any Condor job.
The \SubmitCmd{universe} is \SubmitCmd{grid}.
\SubmitCmd{grid\_resource} specifies the remote \Condor{schedd} daemon to which
the job should be submitted, and its value consists of three fields.
The first field is the grid type, which is \SubmitCmd{condor}.
The second field is the name of the remote \Condor{schedd} daemon.
Its value is the
same as the \Condor{schedd} ClassAd attribute \Attr{Name} on the
remote machine.
The third field is the name of the remote pool's \Condor{collector}.
%and the \SubmitCmd{grid\_type} is \SubmitCmd{condor}. 
%The remote pool's job queue is defined and located by
%the submit description file command \SubmitCmd{remote\_schedd}.
%The value assigned for this command is the same as
%the \Condor{schedd} ClassAd attribute
%\Attr{Name} on the remote machine.
%The remote pool's \Condor{collector} is defined by the submit
%description file command \SubmitCmd{remote\_pool}.

The following represents a minimal submit description file for
a job.

\footnotesize
\begin{verbatim}
# minimal submit description file for a Condor-C job
universe = grid
executable = myjob
output = myoutput
error = myerror
log = mylog

grid_resource = condor joe@remotemachine.example.com remotecentralmanager.example.com
+remote_jobuniverse = 5
+remote_requirements = True
+remote_ShouldTransferFiles = "YES"
+remote_WhenToTransferOutput = "ON_EXIT"
queue
\end{verbatim}
\normalsize

The remote machine needs to understand the attributes of the job.
These are specified in the submit description file using the '+'
syntax, followed by the string \SubmitCmd{remote\_}.
At a minimum, this will be the job's \SubmitCmd{universe} and the job's
\SubmitCmd{requirements}.
It is likely that other attributes specific to the
job's \SubmitCmd{universe} (on the remote pool) will also be necessary.
Note that attributes set with '+' are inserted directly into
the job's ClassAd.  
Specify attributes as they 
must appear in the job's ClassAd, not the submit description file. 
For example,
the \SubmitCmd{universe} is specified using an integer assigned for
a job ClassAd \Attr{JobUniverse}.
Similarly, place quotation marks around string 
expressions.
As an example, a submit description file would ordinarily contain
\footnotesize
\begin{verbatim}
when_to_transfer_output = ON_EXIT
\end{verbatim}
\normalsize
This must appear in the Condor-C job submit description file as
\footnotesize
\begin{verbatim}
+remote_WhenToTransferOutput = "ON_EXIT"
\end{verbatim}
\normalsize

For convenience, the specific entries of 
\SubmitCmd{universe}, 
\SubmitCmd{remote\_grid\_resource}, 
\SubmitCmd{globus\_rsl}, and
\SubmitCmd{globus\_xml}
may be specified as \SubmitCmd{remote\_} commands
without the leading '+'. 
Instead of 
\footnotesize
\begin{verbatim}
+remote_universe = 5
\end{verbatim}
\normalsize

the submit description file command may appear as

\footnotesize
\begin{verbatim}
remote_universe = vanilla
\end{verbatim}
\normalsize

Similarly, the command
\footnotesize
\begin{verbatim}
+remote_gridresource = "condor schedd.example.com cm.example.com"
\end{verbatim}
\normalsize

may be given as

\footnotesize
\begin{verbatim}
remote_grid_resource = condor schedd.example.com cm.example.com
\end{verbatim}
\normalsize

% why we need +remote_ShouldTransferFiles, etc.
For the given example,
the job is to be run as a \SubmitCmd{vanilla} 
\SubmitCmd{universe} job at the remote pool.
The (remote pool's) \Condor{schedd} daemon is likely to
place its job queue data on a local disk 
and execute the job on another machine within the pool of machines.
This implies that the file systems for the resulting submit machine
(the machine specified by \SubmitCmd{remote\_schedd}) and
the execute machine (the machine that runs the job) will
\emph{not} be shared.
Thus,
the two inserted ClassAds
\footnotesize
\begin{verbatim}
+remote_ShouldTransferFiles = "YES"
+remote_WhenToTransferOutput = "ON_EXIT"
\end{verbatim}
\normalsize
are used to invoke Condor's file transfer mechanism. 



As Condor-C is a recent addition to Condor,
the universes, associated integer assignments,
and notes about the existence of functionality are given in 
Table~\ref{working-remote-universes}.
The note "untested" implies that
submissions under the given universe have not yet
been throughly tested.
They may already work.

% universes, and whether they work in Condor-C
% NEED TO UPDATE THIS TABLE FOR gt5 and cream
\begin{center}
\begin{table}[hbt]
\begin{tabular}{|l|l|l}
\textbf{Universe Name} & \textbf{Value} & \textbf{Notes}\\ \hline \hline
standard  & 1 & untested \\ \hline
vanilla   & 5 & works well \\ \hline
scheduler & 7 & works well \\ \hline
grid      & 9 & \\
 & grid\_resource is condor & works well \\
 & grid\_resource is cream & untested \\
 & grid\_resource is gt2  & works well \\
 & grid\_resource is gt4 & untested \\ 
 & grid\_resource is gt5 & untested \\ 
 & grid\_resource is nordugrid & untested \\ 
 & grid\_resource is unicore & untested \\
 & grid\_resource is lsf & works well \\
 & grid\_resource is pbs & works well \\ \hline
java & 10 & untested \\ \hline
parallel & 11 & untested \\ \hline
local & 12 & works well \\ \hline
\end{tabular}
\caption{\label{working-remote-universes}Functionality of remote job universes with Condor-C}
\end{table}
\end{center}

For communication between \Condor{schedd} daemons on the submit
and remote machines,
the location of the remote \Condor{schedd} daemon is needed.
This information resides in the \Condor{collector} of the remote
machine's pool.
The third field of the \SubmitCmd{grid\_resource} command in the submit description file
says which \Condor{collector} should be queried for the remote \Condor{schedd}
daemon's location.
An example of this submit command is
\footnotesize
\begin{verbatim}
grid_resource = condor schedd.example.com machine1.example.com
\end{verbatim}
\normalsize
If the remote \Condor{collector} is not listening on the standard port
(9618), then the port it \emph{is} listening on needs to be specified:
\footnotesize
\begin{verbatim}
grid_resource = condor schedd.example.comd machine1.example.com:12345
\end{verbatim}
\normalsize

File transfer of a job's executable, \File{stdin}, \File{stdout}, and
\File{stderr} are automatic.
When other files need to be transferred using Condor's file transfer
mechanism
(see section~\ref{sec:file-transfer} on page~\pageref{sec:file-transfer}),
the mechanism is applied based on the resulting job universe on the
remote machine.


%%%%%%%%%%%%%%%%%%%%%%%%%%%%%%%%%%%%%%%%%%%%%%%%%%%%%%%%%%%%%%%%%%%%%%%%%%%
\subsubsection{\label{sec:Condor-C-CrossPlatform}
Condor-C Jobs Between Differing Platforms}
%%%%%%%%%%%%%%%%%%%%%%%%%%%%%%%%%%%%%%%%%%%%%%%%%%%%%%%%%%%%%%%%%%%%%%%%%%%

Condor-C jobs given to a remote machine running Windows
must specify the Windows domain of the remote machine.
This is accomplished by defining a ClassAd attribute for the job.
Where the Windows domain is different at the submit machine
from the remote machine, the submit description file 
defines the Windows domain of the remote machine with
\begin{verbatim}
  +remote_NTDomain = "DomainAtRemoteMachine"
\end{verbatim}

A Windows machine not part of a domain
defines the Windows domain as the machine name.

%%%%%%%%%%%%%%%%%%%%%%%%%%%%%%%%%%%%%%%%%%%%%%%%%%%%%%%%%%%%%%%%%%%%%%%%%%%
%\subsubsection{\label{sec:Condor-C-Hops}Multiple Hops for Condor-C Jobs}
%%%%%%%%%%%%%%%%%%%%%%%%%%%%%%%%%%%%%%%%%%%%%%%%%%%%%%%%%%%%%%%%%%%%%%%%%%%
%
% \Todo

%%%%%%%%%%%%%%%%%%%%%%%%%%%%%%%%%%%%%%%%%%%%%%%%%%%%%%%%%%%%%%%%%%%%%%%%%%%
\subsubsection{\label{sec:Condor-C-Limits}Current Limitations in Condor-C}
%%%%%%%%%%%%%%%%%%%%%%%%%%%%%%%%%%%%%%%%%%%%%%%%%%%%%%%%%%%%%%%%%%%%%%%%%%%
\index{Condor-C!limitations}
Submitting jobs to run under the grid universe has not yet
been perfected.
The following is a list of known limitations with Condor-C:

\begin{enumerate}
  \item{Authentication methods other than
  \Expr{CLAIMTOBE}, such as \Expr{GSI} and \Expr{KERBEROS}, are 
  untested, and may not yet work.}

\end{enumerate}

\index{Condor-C|)}


