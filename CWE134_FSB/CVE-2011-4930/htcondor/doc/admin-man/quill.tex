%%%%%%%%%%%%%%%%%%%%%%%%%%%%%%%%%%%%%%%
\section{\label{sec:Quill}Quill}
%%%%%%%%%%%%%%%%%%%%%%%%%%%%%%%%%%%%%%%
\index{Quill|(}
\index{Condor daemon!condor\_quill@\Condor{quill}}
\index{daemon!condor\_quill@\Condor{quill}}
\index{condor\_quill daemon}
\index{Condor daemon!condor\_dbmsd@\Condor{dbmsd}}
\index{daemon!condor\_dbmsd@\Condor{dbmsd}}
\index{condor\_dbmsd daemon}
Quill is an optional component of Condor that maintains a mirror 
of Condor operational data
in a relational database.  The \Condor{quill} daemon updates
the data in the relation database, and the \Condor{dbmsd} daemon
maintains the database itself.

As of Condor version 7.5.5,
Quill is distributed only with the source code.
It is not included in the builds of Condor provided by UW,
but it is available as a feature that can be enabled by those who compile
Condor from the source code.
Find the code within the \File{condor\_contrib} directory, 
in the directories \File{condor\_tt} and \File{condor\_dbmsd}.

%%%%%%%%%%%%%%%%%%%%%%%%%%%%%%%%%%%%%%%
\subsection{\label{sec:Quill-Installation}Installation and Configuration}
%%%%%%%%%%%%%%%%%%%%%%%%%%%%%%%%%%%%%%%

Quill uses the \Prog{PostgreSQL} database management system.
Quill uses the \Prog{PostgreSQL} server as its back end
and client library, 
\Prog{libpq} to talk to the server.
We \Bold{strongly recommend} the use of version 
8.2 or later due to its integrated facilities of certain key 
database maintenance tasks, and stronger security features.

Obtain \Prog{PostgreSQL} from

\URL{http://www.postgresql.org/ftp/source/}

Installation instructions are detailed in:
\URL{http://www.postgresql.org/docs/8.2/static/installation.html}

Configure \Prog{PostgreSQL} after installation:

\begin{enumerate}

\item Initialize the database with the \Prog{PostgreSQL} command 
\verb@initdb@.

\item Configure to accept TCP/IP connections.
For \Prog{PostgreSQL} version 8,
use the \Code{listen\_addresses} variable in 
\File{postgresql.conf} file as a guide.
For example,
\Code{listen\_addresses = '*'}
means listen on any IP interface.

\item Configure automatic vacuuming.
Ensure that these variables with these defaults are
commented in and/or set properly in the 
\File{postgresql.conf} configuration file:
\begin{verbatim}
# Turn on/off automatic vacuuming
autovacuum = on

# time between autovacuum runs, in secs
autovacuum_naptime = 60

# min # of tuple updates before vacuum
autovacuum_vacuum_threshold = 1000

# min # of tuple updates before analyze
autovacuum_analyze_threshold = 500

# fraction of rel size before vacuum
autovacuum_vacuum_scale_factor = 0.4 

# fraction of rel size before analyze
autovacuum_analyze_scale_factor = 0.2

# default vacuum cost delay for 
   # autovac, -1 means use 
   # vacuum_cost_delay
autovacuum_vacuum_cost_delay = -1  

# default vacuum cost limit for 
   # autovac, -1 means use
   # vacuum_cost_limit
autovacuum_vacuum_cost_limit = -1   
\end{verbatim}


\item Configure \Prog{PostgreSQL} to accept TCP/IP connections from 
specific hosts.
Modify the \File{pg\_hba.conf} file 
(which usually resides in the \Prog{PostgreSQL} server's data directory).
Access is required by the \Condor{quill} daemon,
as well as the database users
\Username{quillreader} and \Username{quillwriter}.
For example, to give
database users \Username{quillreader} and \Username{quillwriter}
password-enabled access to all databases on current machine from any
machine in the 128.105.0.0/16 subnet, add the following:

\begin{tabular}{llllll}
host&all&quillreader&128.105.0.0&255.255.0.0&md5\\
host&all&quillwriter&128.105.0.0&255.255.0.0&md5
\end{tabular}

Note that in addition to the database specified by
the configuration variable \MacroNI{QUILL\_DB\_NAME},
the \Condor{quill} daemon also needs access to the database
"template1".
In order to create the database in the first place, 
the \Condor{quill} daemon needs to connect to the database.

\item Start the \Prog{PostgreSQL} server service. See the
installation instructions for the appropriate method to start the service at
\URL{http://www.postgresql.org/docs/8.2/static/installation.html}

\item The \Condor{quill} and \Condor{dbmsd} daemons and client tools connect
to the database as users \Username{quillreader} and 
\Username{quillwriter}.
These are database users, not operating system users.
The two types of users are quite different from each other.
If these database users do not exist,
add them using the 
\Prog{createuser} command supplied with the installation.
Assign them with appropriate passwords;
these passwords will be used by the Quill tools to connect
to the database in a secure way.
User \Username{quillreader} should not be allowed to create
more databases nor create more users.
User \Username{quillwriter} should
not be allowed to create more users,
however it should be allowed to create more databases.
The following commands create the two users
with the appropriate permissions,
and be ready to enter the corresponding passwords when prompted.

\footnotesize
\begin{verbatim}
/path/to/postgreSQL/bin/directory/createuser quillreader \
	--no-createdb --no-createrole --pwprompt

/path/to/postgreSQL/bin/directory/createuser quillwriter \
	--createdb --no-createrole --pwprompt
\end{verbatim}
\normalsize

Answer ``no'' to the question about the ability for role creation.

\item Create a database for Quill to store data in
with the \verb@createdb@ command. 
Create this database with the \Username{quillwriter} user as the owner.
A sample command to do this is
\footnotesize
\begin{verbatim}
createdb -O quillwriter quill
\end{verbatim}
\normalsize
\verb@quill@ is the database name to use with the \MacroNI{QUILL\_DB\_NAME}
configuration variable.

\item The \Condor{quill} and \Condor{dbmsd} daemons need read and write access
to the database.
They connect as user \Username{quillwriter},
which has owner privileges to the database.
Since this gives all access to the \Username{quillwriter} user,
its password cannot be stored in a public place 
(such as in a ClassAd).
For this reason, the \Username{quillwriter} password is stored
in a file named \File{.pgpass} in the Condor spool directory.
Appropriate protections on this file guarantee secure access to the database.
This file must be created and protected by the site administrator;
if this file does not exist as and where expected, the \Condor{quill}
and \Condor{dbmsd} daemons log an error and exit.
The \File{.pgpass} file contains a single line that
has fields separated by colons and is properly terminated by
an operating system specific newline character (Unix) or CRLF (Windows).
The first field may be either the machine name and fully qualified domain,
or it may be a dotted quad IP address.
This is followed by four fields containing:
the TCP port number, 
the name of the database,
the "quillwriter" user name,
and the password.
The form used in the first field must exactly match the value set for 
the configuration variable \Macro{QUILL\_DB\_IP\_ADDR}.
Condor uses a string comparison between the two, and it does not resolve the
host names to compare IP addresses.
Example:
\footnotesize
\begin{verbatim}
machinename.cs.wisc.edu:5432:quill:quillwriter:password
\end{verbatim}
\normalsize

\end{enumerate}

After the \Prog{PostgreSQL} database is initialized and running, 
the Quill schema
must be loaded into it.  First, load the plsql programming language
into the server:

\begin{verbatim}
createlang plpgsql [databasename]
\end{verbatim}

Then, load the Quill schema from the sql files in the \File{sql} subdirectory
of the Condor release directory:

\begin{verbatim}
psql [databasename] [username] < common_createddl.sql
psql [databasename] [username] < pgsql_createddl.sql
\end{verbatim}
where \verb@[username]@ will be \verb@quillwriter@.


After \Prog{PostgreSQL} is configured and running, Condor must also be
configured to use Quill, since by default Quill is configured to be off.

\begin{description}
\item Add the file \File{.pgpass} to the 
  \MacroNI{VALID\_SPOOL\_FILES} variable, since \Condor{preen} must
  be told not to delete this file.
  This step may not be necessary, depending on which version of Condor 
  you are upgrading from. 
 
\item Set up configuration variables that are specific to the installation,
and check that the \MacroNI{HISTORY} variable is set.
\footnotesize
\begin{verbatim}
HISTORY                 = $(SPOOL)/history
QUILL_ENABLED           = TRUE
QUILL_USE_SQL_LOG       = FALSE
QUILL_NAME              = some-unique-quill-name.cs.wisc.edu
QUILL_DB_USER           = quillwriter
QUILL_DB_NAME           = database-for-some-unique-quill-name
QUILL_DB_IP_ADDR        = databaseIPaddress:port
# the following parameter's units is in seconds
QUILL_POLLING_PERIOD    = 10
QUILL_HISTORY_DURATION 	= 30
QUILL_MANAGE_VACUUM     = FALSE
QUILL_IS_REMOTELY_QUERYABLE = TRUE
QUILL_DB_QUERY_PASSWORD =  password-for-database-user-quillreader
QUILL_ADDRESS_FILE      = $(LOG)/.quill_address
QUILL_DB_TYPE           = PGSQL
# The Purge and Reindex intervals are in seconds
DATABASE_PURGE_INTERVAL	= 86400
DATABASE_REINDEX_INTERVAL = 86400
# The History durations are all in days 
QUILL_RESOURCE_HISTORY_DURATION  = 7
QUILL_RUN_HISTORY_DURATION = 7
QUILL_JOB_HISTORY_DURATION = 3650
#The DB Size limit is in gigabytes
QUILL_DBSIZE_LIMIT      = 20
QUILL_MAINTAIN_DB_CONN  = TRUE
SCHEDD_SQLLOG           = $(LOG)/schedd_sql.log
SCHEDD_DAEMON_AD_FILE   = $(LOG)/.schedd_classad

\end{verbatim}
\normalsize

\end{description}

The default Condor configuration file should already contain definitions
for \MacroNI{QUILL} and \MacroNI{QUILL\_LOG}.  
When upgrading from a previous version that did not have Quill to
a new one that does, define these two configuration variables.

Only one machine should run the \Condor{dbmsd} daemon.  
On this machine, add it to the \MacroNI{DAEMON\_LIST} configuration variable.
All Quill-enabled machines should also run the \Condor{quill} daemon.
The machine running the \Condor{dbmsd} daemon can also 
run a \Condor{quill} daemon.  An example \MacroNI{DAEMON\_LIST}
for a machine running both daemons,
and acting as both a submit machine and a central manager might 
look like the following:

\footnotesize
\begin{verbatim}
DAEMON_LIST  = MASTER, SCHEDD, COLLECTOR, NEGOTIATOR, DBMSD, QUILL
\end{verbatim}
\normalsize

The \Condor{dbmsd} daemon will need configuration file entries
common to all daemons.
If not already in the configuration file, add the following entries:

\begin{verbatim}
DBMSD = $(SBIN)/condor_dbmsd
DBMSD_ARGS = -f
DBMSD_LOG = $(LOG)/DbmsdLog
MAX_DBMSD_LOG = 10000000
\end{verbatim}

Descriptions of these and other configuration variables are in
section~\ref{sec:Quill-Config-File-Entries}.
Here are further brief details:

\begin{description}

\item[\Macro{QUILL\_DB\_NAME} and \Macro{QUILL\_DB\_IP\_ADDR}]
These two variables are used to determine the location of the database
server that this Quill would talk to, and the name of the database that
it creates.  More than one Quill server can talk to the same database
server.  This can be accomplished by letting all the 
\MacroNI{QUILL\_DB\_IP\_ADDR} values point to the same database server.

\item[\Macro{QUILL\_DB\_USER}] 
This is the \Prog{PostgreSQL} user that Quill will connect as to the database.
We recommend \Username{quillwriter} for this setting. 
There is no default setting for \MacroNI{QUILL\_DB\_USER}, so it must
be specified in the configuration file. 

\item[\Macro{QUILL\_NAME}]
Each \Condor{quill} daemon in the pool has to be uniquely named.

\item[\Macro{QUILL\_POLLING\_PERIOD}]
This controls the frequency with which Quill polls the
\File{job\_queue.log} file.  By default, it is 10 seconds.  Since Quill
works by periodically sniffing the log file for updates and then sending
those updates to the database, this variable controls the trade off between
the currency of query results and Quill's load on the system, which
is usually negligible.

\item[ \Macro{QUILL\_RESOURCE\_HISTORY\_DURATION}, 
		\Macro{QUILL\_RUN\_HISTORY\_DURATION}, and
		\Macro{QUILL\_JOB\_HISTORY\_DURATION}]
These three variables control the deletion of historical information from the
database. 
\MacroNI{QUILL\_RESOURCE\_HISTORY\_DURATION} is the number of days
historical information about the state of a resource will be kept in the 
database. 
The default for resource history is 7 days.
An example of a resource is the ClassAd for a compute slot.
\MacroNI{QUILL\_RUN\_HISTORY\_DURATION} is the number of days
after completion that auxiliary information about a  given job will stay in 
the database.  
This includes user log events, file transfers performed by the job, the matches
that were made for a job, et cetera. 
The default for run history is 7 days.
\MacroNI{QUILL\_JOB\_HISTORY\_DURATION} is the number of days
after completion that a given job will stay in the database.  
A more precise definition is the number of days since the history ad got 
into the history database; those two might be different,
if a job is completed but stays in the queue for a while.
The default for job history is 3,650 days (about 10 years.)

\item[\Macro{DATABASE\_PURGE\_INTERVAL}]
As scanning the entire database for old jobs can be expensive,
the other variable \MacroNI{DATABASE\_PURGE\_INTERVAL}
is the number of seconds between two successive scans.  
\MacroNI{DATABASE\_PURGE\_INTERVAL} is set to 86400 seconds, or one day.

\item[\Macro{DATABASE\_REINDEX\_INTERVAL}]
\Prog{PostgreSQL} does not aggressively maintain the index structures for
deleted tuples. This can lead to bloated index structures.
Quill can periodically reindex the database, which is controlled by the
variable \MacroNI{DATABASE\_REINDEX\_INTERVAL}.
\MacroNI{DATABASE\_PURGE\_INTERVAL} is set to 86400 seconds, or one day.

\item[\Macro{QUILL\_DBSIZE\_LIMIT}]
Quill can estimate the size of the database, and send email to the Condor
administrator if the database size exceeds this threshold. 
The estimate is checked after every \MacroNI{DATABASE\_PURGE\_INTERVAL}.
The limit is given as gigabytes, and the default is 20.

\item[\Macro{QUILL\_MAINTAIN\_DB\_CONN}]
Quill can maintain an open connection the database server, which speeds up
updates to the database. 
However, each open connection consumes resources at the database server.
The default is TRUE, but for large pools we recommend setting this 
FALSE.

\item[\Macro{QUILL\_MANAGE\_VACUUM}]
Set to \Expr{False} by default, this variable determines whether Quill is to 
perform vacuuming tasks on its tables or not. Vacuuming is a maintenance 
task that needs to be performed on tables in \Prog{PostgreSQL}. The 
frequency with which a table is vacuumed typically depends on the number 
of updates (inserts/deletes) performed on the table. Fortunately, with 
\Prog{PostgreSQL} version 8.1, vacuuming tasks can be configured to be 
performed automatically by the database server. We recommend that users 
upgrade to 8.1 and use the integrated vacuuming facilities of the database 
server, instead of having Quill do them. If the user does prefer having 
Quill perform those vacuuming tasks, it can be achieved by setting this 
variable to Expr{True}. However, it cannot be overstated that Quill's vacuuming 
policy is quite rudimentary as compared to the integrated facilities 
of the database server, and under high update workloads, can prove to 
be a bottleneck on the Quill daemon. As such, setting this variable to 
Expr{True} results in some warning messages in the log file regarding this 
issue.

\item[\Macro{QUILL\_IS\_REMOTELY\_QUERYABLE}]
Thanks to 
\Prog{PostgreSQL},
one can now remotely query both the job queue and the
history tables. This variable controls whether this remote querying 
feature should be enabled.  By default it is \Expr{True}.  Note that even if 
this is \Expr{False}, one can still query the job queue 
at the remote \Condor{schedd} daemon.
This variable only controls whether the database tables are remotely queryable.

\item[\Macro{QUILL\_DB\_QUERY\_PASSWORD}]
In order for the query tools to connect to a database, they need to provide
the password that is assigned to the database user \Username{quillreader}. 
This variable is then advertised by the \Condor{quill} daemon
to the \Condor{collector}.  
This facility enables remote querying: remote \Condor{q} query tools first 
ask the \Condor{collector} for
the password associated with a particular Quill database, 
and then query that database.  Users who do not have access to the 
\Condor{collector} 
cannot view the password, and as such cannot query the database.  Again, this 
password only provides read access to the database.

\item[\Macro{QUILL\_ADDRESS\_FILE}]
When Quill starts up, it can place its address (IP and port)
into a file.  This way, tools running on the local machine do not
need to query the central manager to find Quill.  This 
feature can be turned off by commenting out the variable.

\end{description}


%%%%%%%%%%%%%%%%%%%%%%%%%%%%%%%%%%%%%%%
\subsection{\label{sec:Quill-Example}Four Usage Examples}
%%%%%%%%%%%%%%%%%%%%%%%%%%%%%%%%%%%%%%%


\begin{enumerate}
\item Query a remote Quill daemon on \File{regular.cs.wisc.edu}
for all the jobs in the queue
\begin{verbatim}
	condor_q -name quill@regular.cs.wisc.edu
	condor_q -name schedd@regular.cs.wisc.edu

\end{verbatim}
There are two ways to get to a Quill daemon: directly using its name as 
specified in the \MacroNI{QUILL\_NAME} configuration variable, or indirectly
by querying the \Condor{schedd} daemon using its name.
In the latter case, \Condor{q} will detect 
if that \Condor{schedd} daemon is being serviced by a database, and if so, directly query it.
In both cases, the IP address and port of the database server hosting the data of 
this particular remote Quill daemon can be figured out by the \MacroNI{QUILL\_DB\_IP\_ADDR} 
and \MacroNI{QUILL\_DB\_NAME} variables specified in the \MacroNI{QUILL\_AD}
sent by the quill daemon to the collector and in the \MacroNI{SCHEDD\_AD} sent by
the \Condor{schedd} daemon.  

\item Query a remote Quill daemon on \File{regular.cs.wisc.edu} for all historical 
jobs belonging to owner einstein.
\begin{verbatim}
	condor_history -name quill@regular.cs.wisc.edu einstein
\end{verbatim}

\item Query the local Quill daemon for the average time spent in the queue 
for all non-completed jobs. 
\begin{verbatim}
	condor_q -avgqueuetime 
\end{verbatim}
The average queue time is defined as the average of
\Expr{(currenttime - jobsubmissiontime)} over all jobs which are neither
completed \Expr{(JobStatus == 4)} or removed \Expr{(JobStatus == 3)}.

\item Query the local Quill daemon for all historical jobs completed since 
Apr 1, 2005 at 13h 00m.
\begin{verbatim}
	condor_history -completedsince '04/01/2005 13:00'
\end{verbatim}
It fetches all jobs
which got into the 'Completed' state on or after the
specified time stamp.  It use the \Prog{PostgreSQL} date/time
syntax rules, as it encompasses most format options.  See
\URL{http://www.postgresql.org/docs/8.2/static/datatype-datetime.html}
for the various time stamp formats.

\end{enumerate}

%%%%%%%%%%%%%%%%%%%%%%%%%%%%%%%%%%%%%%%
\subsection{\label{sec:Quill-Security}Quill and Security}
%%%%%%%%%%%%%%%%%%%%%%%%%%%%%%%%%%%%%%%

There are several layers of security in Quill, some provided by Condor and
others provided by the database.  First, all accesses to the database
are password-protected.

\begin{enumerate}
\item The query tools, \Condor{q} and
\Condor{history} connect to the database as user \Username{quillreader}.
The password for this user can vary from one database to another and
as such, each Quill daemon advertises this password to the collector.
The query tools then obtain this password from the collector and
connect successfully to the database.  Access to the database by the
\Username{quillreader} user is read-only, as this is sufficient for the
query tools.  The \Condor{quill} daemon ensures this protected access using the sql
GRANT command when it first creates the tables in the database.  Note that
access to the \Username{quillreader} password itself can be blocked by
blocking access to the collector, a feature already supported in Condor.

\item The \Condor{quill} and \Condor{dbmsd} daemons, on the other hand,
need read and write access to the database.
As such, they connect as user \Username{quillwriter},
who has owner privileges to the database.  Since this gives all
access to the \Username{quillwriter} user, this password cannot
be stored in a public place (such as the collector).  For this
reason, the \Username{quillwriter} password is stored in a file called
\File{.pgpass} in the Condor spool directory.
Appropriate protections on this file guarantee secure access to the database.
This file must be created and protected by the site administrator;
if this file does not exist as and where expected, the \Condor{quill}
daemon logs an error and exits.

\item The \Attr{IsRemotelyQueryable} attribute in the Quill ClassAd advertised
by the Quill daemon to the collector can be used by site administrators
to disallow the database from being read by all remote Condor query tools.

\end{enumerate}

%%%%%%%%%%%%%%%%%%%%%%%%%%%%%%%%%%%%%%%
\subsection{\label{sec:Quill-Schema}Quill and Its RDBMS Schema}
%%%%%%%%%%%%%%%%%%%%%%%%%%%%%%%%%%%%%%%

\newcommand{\cd}{{C}ondor}
\newcommand{\qp}{{Q}uill}
\newcommand{\ca}{{C}lass{A}d}
\newcommand{\cas}{{C}lass{A}ds}
\newcommand{\pg}{{P}ostgre{SQL}}
\newcommand{\oc}{{O}racle}
\newcommand{\db}{{D}bmsd}

\noindent \textbf{Notes:}
\begin{itemize}
\item The type ``timestamp(\textit{precision}) with timezone'' is
abbreviated ``ts(\textit{precision}) w tz.''
\item The column O. Type is an abbreviation for Oracle Type.
\item The column P. Type is an abbreviation for {PostgreSQL} Type.
\end{itemize}
Although the current version of Condor does not support Oracle, we
anticipate supporting it in the future, so Oracle support in this
schema document is for future reference.
\vspace{24pt}

%%% ADMINISTRATIVE TABLES
\subsubsection{Administrative Tables}
\begin{center}
% currencies
  \begin{tabular}{|l|l|l|p{3.3in}|}\hline
    \multicolumn{4}{|c|}{\textbf{Attributes of currencies Table}}\\ \hline
    \textbf{Name} & \textbf{O. Type} & \textbf{P. Type} & \textbf{Description}\\ \hline
    datasource & varchar(4000) & varchar(4000) & Identifier of the data source.\\ \hline
    lastupdate & ts(3) w tz & ts(3) w tz & Time of the last update sent to the database from the data source.\\ \hline
  \end{tabular}
\vspace{24pt}

% error_sqllogs
  \begin{tabular}{|l|l|l|p{3.2in}|}\hline
    \multicolumn{4}{|c|}{\textbf{Attributes of error\_sqllogs Table}}\\ \hline
    \textbf{Name} & \textbf{O. Type} & \textbf{P. Type} & \textbf{Description}\\ \hline
    logname & varchar(100) & varchar(100) & Name of the {SQL} log file causing a {SQL} error. \\ \hline
    host & varchar(50) & varchar(50) & The host where the {SQL} log resides.\\ \hline
    lastmodified & ts(3) w tz & ts(3) w tz & The last modified time of the {SQL} log. \\ \hline
    errorsql & varchar(4000) & text & The {SQL} statement causing an error.\\ \hline
    logbody & clob & text & The body of the {SQL} log.\\ \hline
    errormessage & varchar(4000) & varchar(4000) & The description of the error.\\
    \multicolumn{4}{|l|}{\textbf{INDEX:} Index named error\_sqllog\_idx on (logname, host, lastmodified)}\\ \hline
  \end{tabular}
\vspace{24pt}

% maintenance_log
  \begin{tabular}{|l|l|l|p{3.4in}|}\hline
    \multicolumn{4}{|c|}{\textbf{Attributes of maintenance\_log Table}}\\ \hline
    \textbf{Name} & \textbf{O. Type} & \textbf{P. Type} & \textbf{Description}\\ \hline
    eventts & ts(3) w tz & ts(3) w tz & Time the event occurred.\\ \hline
    eventmsg & varchar(4000) & varchar(4000) & Message describing the event.\\ \hline
  \end{tabular}
\vspace{24pt}

% quilldbmonitor
  \begin{tabular}{|l|l|l|p{4.1in}|}\hline
    \multicolumn{4}{|c|}{\textbf{Attributes of quilldbmonitor Table}}\\ \hline
    \textbf{Name} & \textbf{O. Type} & \textbf{P. Type} & \textbf{Description}\\ \hline
    dbsize & integer & integer & Size of the database in megabytes.\\ \hline
  \end{tabular}
\vspace{24pt}

% quill_schema_version
  \begin{tabular}{|l|l|l|p{3.6in}|}\hline
    \multicolumn{4}{|c|}{\textbf{Attributes of quill\_schema\_version Table}}\\ \hline
    \textbf{Name} & \textbf{O. Type} & \textbf{P. Type} & \textbf{Description}\\ \hline
    major & int & int & Major version number. \\ \hline
    minor & int & int & Minor version number. \\ \hline
    back\_to\_major & int & int & The major number of the old version this version is compatible to. \\ \hline
    back\_to\_minor & int & int & The minor number of the old version this version is compatible to. \\ \hline
  \end{tabular}
\vspace{24pt}

% throwns
  \begin{tabular}{|l|l|l|p{3.2in}|}\hline
    \multicolumn{4}{|c|}{\textbf{Attributes of throwns Table}}\\ \hline
    \textbf{Name} & \textbf{O. Type} & \textbf{P. Type} & \textbf{Description}\\ \hline
    filename & varchar(4000) & varchar(4000) & The name of the log that was truncated. \\ \hline
    machine\_id & varchar(4000) & varchar(4000) & The machine where the truncated log resides. \\ \hline
    log\_size & numeric(38) & numeric(38) & The size of the truncated log. \\ \hline
    throwtime & ts(3) w tz & ts(3) w tz & The time when the truncation occurred. \\ \hline
  \end{tabular}
\vspace{24pt}
\end{center}


%%% DAEMON TABLES
\subsubsection{Daemon Tables}
\begin{center}
% daemons_horizontal
  \begin{tabular}{|l|l|l|p{2.3in}|}\hline
    \multicolumn{4}{|c|}{\textbf{Attributes of daemons\_horizontal Table}}\\ \hline
    \textbf{Name} & \textbf{O. Type} & \textbf{P. Type} & \textbf{Description}\\ \hline
    mytype & varchar(100) & varchar(100) & The type of daemon \ca, e.g. ``Master'' \\ \hline
    name & varchar(500) & varchar(500) & The name identifier of the daemon \ca. \\ \hline
    lastreportedtime & ts(3) w tz & ts(3) w tz & Time when the daemon last reported to \qp. \\ \hline
    monitorselftime & ts(3) w tz & ts(3) w tz & The time when the daemon last collected information about itself. \\ \hline
    monitorselfcpuusage & numeric(38) & numeric(38) & The amount of {CPU} this daemon has used. \\ \hline
    monitorselfimagesize & numeric(38) & numeric(38) & The amount of virtual memory this daemon has used. \\ \hline
    monitorselfresidentsetsize & numeric(38) & numeric(38) & The amount of physical memory this daemon has used. \\ \hline
    monitorselfage & integer & integer & How long the daemon has been running. \\ \hline
    updatesequencenumber & integer & integer & The sequence number associated with the update. \\ \hline
    updatestotal & integer & integer & The number of updates received from the daemon. \\ \hline
    updatessequenced & integer & integer & The number of updates that were in order. \\ \hline
    updateslost & integer & integer & The number of updates that were lost. \\ \hline
    updateshistory & varchar(4000) & varchar(4000) & Bitmask of the last 32 updates. \\ \hline
    lastreportedtime\_epoch & integer & integer & The equivalent epoch time of last heard from. \\ \hline
    \multicolumn{4}{|l|}{\textbf{PRIMARY KEY:} (mytype, name)} \\ \hline
    \multicolumn{4}{|l|}{\textbf{NOT NULL:} mytype and name cannot be null} \\ \hline
  \end{tabular}
\vspace{24pt}

% daemons_horizontal_history
  \begin{tabular}{|l|l|l|p{2.3in}|}\hline
    \multicolumn{4}{|c|}{\textbf{Attributes of daemons\_horizontal\_history Table}}\\ \hline
    \textbf{Name} & \textbf{O. Type} & \textbf{P. Type} & \textbf{Description}\\ \hline
    mytype & varchar(100) & varchar(100) & The type of daemon \ca, e.g. ``Master'' \\ \hline
    name & varchar(500) & varchar(500) & The name identifier of the daemon \ca. \\ \hline
    lastreportedtime & ts(3) w tz & ts(3) w tz &  Time when the daemon last reported to \qp. \\ \hline
    monitorselftime & ts(3) w tz & ts(3) w tz & The time when the daemon last collected information about itself. \\ \hline
    monitorselfcpuusage & numeric(38) & numeric(38) & The amount of {CPU} this daemon has used. \\ \hline
    monitorselfimagesize & numeric(38) & numeric(38) & The amount of virtual memory this daemon has used. \\ \hline
    monitorselfresidentsetsize & numeric(38) & numeric(38) & The amount of physical memory this daemon has used. \\ \hline
    monitorselfage & integer & integer & How long the daemon has been running. \\ \hline
    updatesequencenumber & integer & integer & The sequence number associated with the update. \\ \hline
    updatestotal & integer & integer & The number of updates received from the daemon. \\ \hline
    updatessequenced & integer & integer & The number of updates that were in order. \\ \hline
    updateslost & integer & integer & The number of updates that were lost. \\ \hline
    updateshistory & varchar(4000) & varchar(4000) &  Bitmask of the last 32 updates. \\ \hline
    endtime & ts(3) w tz & ts(3) w tz & End of when the \ca\ is valid. \\ \hline
  \end{tabular}
\vspace{24pt}

% daemons_vertical
  \begin{tabular}{|l|l|l|p{3.1in}|}\hline
    \multicolumn{4}{|c|}{\textbf{Attributes of daemons\_vertical Table}}\\ \hline
    \textbf{Name} & \textbf{O. Type} & \textbf{P. Type} & \textbf{Description}\\ \hline
    mytype & varchar(100) & varchar(100) & The type of daemon \ca, e.g. ``Master'' \\ \hline
    name & varchar(500) & varchar(500) &  The name identifier of the daemon \ca. \\ \hline
    attr & varchar(4000) & varchar(4000) & Attribute name.\\ \hline
    val & clob & text & Attribute value.\\ \hline
    lastreportedtime & ts(3) w tz & ts(3) w tz & Time when the daemon last reported to \qp.\\ \hline
    \multicolumn{4}{|l|}{\textbf{PRIMARY KEY:} (mytype, name, attr)} \\ \hline
    \multicolumn{4}{|l|}{\textbf{NOT NULL:} mytype, name, and attr cannot be null} \\ \hline
  \end{tabular}
\vspace{24pt}

% daemons_vertical_history
  \begin{tabular}{|l|l|l|p{3.1in}|}\hline
    \multicolumn{4}{|c|}{\textbf{Attributes of daemons\_vertical\_history Table}}\\ \hline
    \textbf{Name} & \textbf{O. Type} & \textbf{P. Type} & \textbf{Description}\\ \hline
    mytype & varchar(100) & varchar(100) & The type of daemon \ca, e.g. ``Master'' \\ \hline
    name & varchar(500) & varchar(500) & The name identifier of the daemon \ca. \\ \hline
    lastreportedtime & ts(3) w tz & ts(3) w tz &  Time when the daemon last reported to \qp.\\ \hline
    attr & varchar(4000) & varchar(4000) & Attribute name.\\ \hline
    val & clob & text & Attribute value.\\ \hline
    endtime & ts(3) w tz & ts(3) w tz & End of when the \ca\ is valid. \\ \hline
  \end{tabular}
\vspace{24pt}

% submitters_horizontal
  \begin{tabular}{|l|l|l|p{3.1in}|}\hline
    \multicolumn{4}{|c|}{\textbf{Attributes of submitters\_horizontal table}}\\ \hline
    \textbf{Name} & \textbf{O. Type} & \textbf{P. Type} & \textbf{Description}\\ \hline
    name & varchar(500) & varchar(500) & Name of the submitter \ca. \\ \hline
    scheddname & varchar(4000) & varchar(4000) & Name of the schedd where the submitter ad is from. \\ \hline
    lastreportedtime & ts(3) w tz & ts(3) w tz & Last time a submitter \ca\ was sent to \qp. \\ \hline
    idlejobs & integer & integer & Number of idle jobs of the submitter. \\ \hline
    runningjobs & integer & integer & Number of running jobs of the submitter. \\ \hline
    heldjobs & integer & integer & Number of held jobs of the submitter. \\ \hline
    flockedjobs & integer & integer & Number of flocked jobs of the submitter. \\ \hline
  \end{tabular}
\vspace{24pt}

% submitters_horizontal_history
  \begin{tabular}{|l|l|l|p{3.1in}|}\hline
    \multicolumn{4}{|c|}{\textbf{Attributes of submitters\_horizontal\_history table}}\\ \hline
    \textbf{Name} & \textbf{O. Type} & \textbf{P. Type} & \textbf{Description}\\ \hline
    name & varchar(500) & varchar(500) & Name of the submitter \ca. \\ \hline
    scheddname & varchar(4000) & varchar(4000) & Name of the schedd where the submitter ad is from. \\ \hline
    lastreportedtime & ts(3) w tz & ts(3) w tz & Last time a submitter \ca\ was sent to \qp. \\ \hline
    idlejobs & integer & integer & Number of idle jobs of the submitter. \\ \hline
    runningjobs & integer & integer & Number of running jobs of the submitter. \\ \hline
    heldjobs & integer & integer & Number of held jobs of the submitter. \\ \hline
    flockedjobs & integer & integer & Number of flocked jobs of the submitter. \\ \hline
    endtime & ts(3) w tz & ts(3) w tz & End of when the \ca\ is valid. \\ \hline
  \end{tabular}
\vspace{24pt}
\end{center}


%%% FILES TABLES
\subsubsection{Files Tables}
\begin{center}
% files
  \begin{tabular}{|l|l|l|p{3.2in}|}\hline
    \multicolumn{4}{|c|}{\textbf{Attributes of files Table}}\\ \hline
    \textbf{Name} & \textbf{O. Type} & \textbf{P. Type} & \textbf{Description}\\ \hline
    file\_id & int & int & Unique numeric identifier of the file.\\ \hline
    name & varchar(4000) & varchar(4000) & File name.\\ \hline
    host & varchar(4000) & varchar(4000) & Name of machine where the file is located.\\ \hline
    path & varchar(4000) & varchar(4000) & Directory path to the file.\\ \hline
    acl\_id & integer & integer & Not yet used, null.\\ \hline
    lastmodified & ts(3) w tz & ts(3) w tz & Timestamp of the file.\\ \hline
    filesize & numeric(38) & numeric(38) & Size of the file in bytes.\\ \hline
    checksum & varchar(32) & varchar(32) & MD5 checksum of the file.\\ \hline
    \multicolumn{4}{|l|}{\textbf{PRIMARY KEY:} file\_id} \\ \hline
    \multicolumn{4}{|l|}{\textbf{NOT NULL:} file\_id cannot be null} \\ \hline
  \end{tabular}
\vspace{24pt}

% fileusages
  \begin{tabular}{|l|l|l|p{3.2in}|}\hline
    \multicolumn{4}{|c|}{\textbf{Attributes of fileusages Table}}\\ \hline
    \textbf{Name} & \textbf{O. Type} & \textbf{P. Type} & \textbf{Description}\\ \hline
    globaljobid & varchar(4000) & varchar(4000) & Global identifier of the job that used the file.\\ \hline
    file\_id & int & int & Numeric identifier of the file.\\ \hline
    usagetype & varchar(4000) & varchar(4000) & Type of use of the file by the job, e.g., input, output, command.\\ \hline
    \multicolumn{4}{|l|}{\textbf{REFERENCE:} file\_id references files(file\_id)} \\ \hline
  \end{tabular}
\vspace{24pt}

% transfers
  \begin{tabular}{|l|l|l|p{2.4in}|}\hline
    \multicolumn{4}{|c|}{\textbf{Attributes of transfers Table}}\\ \hline
    \textbf{Name} & \textbf{O. Type} & \textbf{P. Type} & \textbf{Description}\\ \hline
    globaljobid & varchar(4000) & varchar(4000) & Unique global identifier for the job.\\ \hline
    src\_name & varchar(4000) & varchar(4000) & Name of the file on the source machine.\\ \hline
    src\_host & varchar(4000) & varchar(4000) & Name of the source machine.\\ \hline
    src\_port & integer & integer & Source port number used for the transfer.\\ \hline
    src\_path & varchar(4000) & varchar(4000) & Path to the file on the source machine.\\ \hline
    src\_daemon & varchar(30) & varchar(30) & \cd\ demon performing the transfer on the source machine.\\ \hline
    src\_protocol & varchar(30) & varchar(30) & The protocol used on the source machine.\\ \hline
    src\_credential\_id & integer & integer & Not yet used, null.\\ \hline
    src\_acl\_id & integer & integer & Not yet used, null.\\ \hline
    dst\_name & varchar(4000) & varchar(4000) & Name of the file on the destination machine.\\ \hline
    dst\_host & varchar(4000) & varchar(4000) & Name of the destination machine.\\ \hline
    dst\_port & integer & integer & Destination port number used for the transfer.\\ \hline
    dst\_path & varchar(4000) & varchar(4000) & Path to the file on the destination machine.\\ \hline
    dst\_daemon & varchar(30) & varchar(30) & \cd\ daemon receiving the transfer on the destination machine.\\ \hline
    dst\_protocol & varchar(30) & varchar(30) & The protocol used on the destination machine.\\ \hline
    dst\_credential\_id & integer & integer & Not yet used, null.\\ \hline
    dst\_acl\_id & integer & integer &  Not yet used, null.\\ \hline
    transfer\_intermediary\_id & integer & integer &  Not yet used, null; will use someday if a proxy is used.\\ \hline
    transfer\_size\_bytes & numeric(38) & numeric (38) & Size of the data transfered in bytes.\\ \hline
    elapsed & numeric(38) & numeric(38) & Number of seconds that elapsed during the transfer.\\ \hline
    checksum & varchar(256) & varchar(256) & Checksum of the file.\\ \hline
    transfer\_time & ts(3) w tz & ts(3) w tz & Time when the transfer took place.\\ \hline
    last\_modified & ts(3) w tz & ts(3) w tz & Last modified time for the file that was transfered.\\ \hline
    is\_encrypted & varchar(5) & varchar(5) & (boolean) True if the file is encrypted.\\ \hline
    delegation\_method\_id & integer & integer & Not yet used, null.\\ \hline
    completion\_code & integer & integer & Indicates whether the transfer failed or succeeded.\\ \hline
  \end{tabular}
\vspace{24pt}
\end{center}


%%% INTERFACE TABLES
\subsubsection{Interface Tables}
\begin{center}
% cdb_users
  \begin{tabular}{|l|l|l|p{3.4in}|}\hline
    \multicolumn{4}{|c|}{\textbf{Attributes of cdb\_users Table}}\\ \hline
    \textbf{Name} & \textbf{O. Type} & \textbf{P. Type} & \textbf{Description}\\ \hline
    userid & varchar(30) & varchar(30) & Unique identifier of the user\\ \hline
    password & character(32) & character(32) & Encrypted password\\ \hline
    admin & varchar(5) & varchar(5) & (boolean) True if the user has administrator privileges\\ \hline
  \end{tabular}\\
\vspace{24pt}

% l_eventtype
  \begin{tabular}{|l|l|l|p{3.2in}|}\hline
    \multicolumn{4}{|c|}{\textbf{Attributes of l\_eventtype Table}}\\ \hline
    \textbf{Name} & \textbf{O. Type} & \textbf{P. Type} & \textbf{Description}\\ \hline
    eventtype & integer & integer & Numeric type code of the event.\\ \hline
    description & varchar(4000) & varchar(4000) & Description of the type of event associated with the eventtype code.\\ \hline
  \end{tabular}
\vspace{24pt}

% l_jobstatus
  \begin{tabular}{|l|l|l|p{3.2in}|}\hline
    \multicolumn{4}{|c|}{\textbf{Attributes of l\_jobstatus Table}}\\ \hline
    \textbf{Name} & \textbf{O. Type} & \textbf{P. Type} & \textbf{Description}\\ \hline
    jobstatus & integer & integer & Numeric code for job status.\\ \hline
    abbrev & char(1) & char(1) & Single letter code for job status.\\ \hline
    description & varchar(4000) & varchar(4000) & Description of job status.\\ \hline
    \multicolumn{4}{|l|}{\textbf{PRIMARY KEY:} jobstatus} \\ \hline
    \multicolumn{4}{|l|}{\textbf{NOT NULL:} jobstatus cannot be null} \\ \hline
  \end{tabular}
\vspace{24pt}
\end{center}


%%% JOBS TABLES
\subsubsection{Jobs Tables}
\begin{center}
% clusterads_horizontal
  \begin{tabular}{|l|l|l|p{2.6in}|}\hline
    \multicolumn{4}{|c|}{\textbf{Attributes of clusterads\_horizontal Table}}\\ \hline
    \textbf{Name} & \textbf{O. Type} & \textbf{P. Type} & \textbf{Description}\\ \hline
    scheddname & varchar(4000) & varchar(4000) & Name of the schedd the job is submitted to.\\ \hline
    cluster\_id & integer & integer & Cluster identifier for the job.\\ \hline
    owner & varchar(30) & varchar(30) & User who submitted the job.\\ \hline
    jobstatus & integer & integer & Current status of the job.\\ \hline
    jobprio & integer & integer & Priority for this job.\\ \hline
    imagesize & numeric(38) & numeric(38) & Estimate of memory image size of the job in kilobytes.\\ \hline
    qdate & ts(3) w tz & ts(3) w tz & Time the job was submitted to the job queue.\\ \hline
    remoteusercpu & numeric(38) & numeric(38) & Total number of seconds of user CPU time the job used on remote machines.\\ \hline
    remotewallclocktime & numeric(38) & numeric(38) & Committed cumulative number of seconds the job has been allocated to a machine.\\ \hline
    cmd & clob & text & Path to and filename of the job to be executed.\\ \hline
    args & clob & text & Arguments passed to the job.\\ \hline
    jobuniverse & integer & integer & The \cd\ universe used by the job.\\ \hline
    \multicolumn{4}{|l|}{\textbf{PRIMARY KEY:} (scheddname, cluster\_id)} \\ \hline
    \multicolumn{4}{|l|}{\textbf{NOT NULL:} scheddname and cluster\_id cannot be null} \\ \hline
  \end{tabular}
\vspace{24pt}

% clusterads_vertical
\begin{tabular}{|l|l|l|p{3.2in}|}\hline
    \multicolumn{4}{|c|}{\textbf{Attributes of clusterads\_vertical Table}}\\ \hline
    \textbf{Name} & \textbf{O. Type} & \textbf{P. Type} & \textbf{Description}\\ \hline
    scheddname & varchar(4000) & varchar(4000) & Name of the schedd that the job is submitted to.\\ \hline
    cluster\_id & integer & integer & Cluster identifier for the job.\\ \hline
    attr & varchar(2000) & varchar(2000) & Attribute name.\\ \hline
    val & clob & text & Attribute value.\\ \hline
    \multicolumn{4}{|l|}{\textbf{PRIMARY KEY:} (scheddname, cluster\_id, attr)} \\ \hline
  \end{tabular}
\vspace{24pt}

% jobs_horizontal_history Part 1
  \begin{tabular}{|l|l|l|p{2.4in}|}\hline
    \multicolumn{4}{|c|}{\textbf{Attributes of jobs\_horizontal\_history Table -- Part 1 of 3}}
\\ \hline
    \textbf{Name} & \textbf{O. Type} & \textbf{P. Type} & \textbf{Description}\\ \hline
    scheddname & varchar(4000) & varchar(4000) & Name of the schedd that submitted the job.\\ \hline
    scheddbirthdate & integer & integer & The birth date of the schedd where the job is submitted.\\ \hline
    cluster\_id & integer & integer & Cluster identifier for the job.\\ \hline
    proc\_id & integer & integer & Process identifier for the job.\\ \hline
    qdate & ts(3) w tz & ts(3) w tz & Time the job was submitted to the job queue.\\ \hline
    owner & varchar(30) & varchar(30) & User who submitted the job.\\ \hline
    globaljobid & varchar(4000) & varchar(4000) & Unique global identifier for the job.\\ \hline
    numckpts & integer & integer & Number of checkpoints written by the job during its lifetime.\\ 
\hline
    numrestarts & integer & integer & Number of restarts from a checkpoint attempted by the job in its lifetime.\\ \hline
    numsystemholds & integer & integer & Number of times Condor-G placed the job on hold.\\ \hline
    condorversion & varchar(4000) & varchar(4000) & Version of \cd\ that ran the job.\\ \hline
    condorplatform & varchar(4000) & varchar(4000) & Platform of the computer where the schedd runs.\\ \hline
    rootdir & varchar(4000) & varchar(4000) & Root directory on the system where the job is submitted from.\\ \hline
    iwd & varchar(4000) & varchar(4000) & Initial working directory of the job.\\ \hline
    jobuniverse & integer & integer & The Condor universe used by the job.\\ \hline
    cmd & clob & text & Path to and filename of the job to be executed.\\ \hline
    minhosts & integer & integer & Minimum number of hosts that must be in the claimed state for this job, before the job may enter the running state.\\ \hline
    maxhosts & integer & integer & Maximum number of hosts this job would like to claim.\\ \hline
    jobprio & integer & integer & Priority for this job.\\ \hline
    negotiation\_user\_name & varchar(4000) & varchar(4000) & User name in which the job is negotiated.\\ \hline
    env & clob & text & Environment under which the job ran.\\ \hline
    userlog & varchar(4000) & varchar(4000) & User log where the job events are written to.\\ \hline
    coresize & numeric(38) & numeric(38) & Maximum allowed size of the core file.\\ \hline
    \multicolumn{4}{|c|}{\textbf{Table Continues on Next Page}}\\ \hline
  \end{tabular}
\vspace{24pt}

% jobs_horizontal_history Part 2
  \begin{tabular}{|l|l|l|p{2.6in}|}\hline
    \multicolumn{4}{|c|}{\textbf{Attributes of jobs\_horizontal\_history Table -- Part 2 of 3}}
\\ \hline
    \textbf{Name} & \textbf{O. Type} & \textbf{P. Type} & \textbf{Description}\\ \hline
    killsig & varchar(4000) & varchar(4000) & Signal to be sent if the job is put on hold.\\ \hline
    stdin & varchar(4000) & varchar(4000) & The file used as stdin.\\ \hline
    transferin & varchar(5) & varchar(5) & (boolean) For globus universe jobs. True if input should be transferred to the remote machine.\\ \hline
    stdout & varchar(4000) & varchar(4000) & The file used as stdout.\\ \hline
    transferout & varchar(5) & varchar(5) & (boolean) For globus universe jobs. True if output should be transferred back to the submit machine.\\ \hline
    stderr & varchar(4000) & varchar(4000) & The file used as stderr.\\ \hline
    transfererr & varchar(5) & varchar(5) & (boolean) For globus universe jobs. True if error output should be transferred back to the submit machine.\\ \hline
    shouldtransferfiles & varchar (4000) & varchar(4000) & Whether \cd\ should transfer files to and from the machine where the job runs.\\ \hline
    transferfiles & varchar(4000) & varchar(4000) & Depreciated. Similar to shouldtransferfiles.\\ \hline
    executablesize & numeric(38) & numeric(38) & Size of the executable in kilobytes.\\ \hline
    diskusage & integer & integer & Size of the executable and input files to be transferred.\\ \hline
    filesystemdomain & varchar(4000) & varchar(4000) & Name of the networked file system used by the job.\\ \hline
    args & clob & text & Arguments passed to the job.\\ \hline
    lastmatchtime & ts(3) w tz & ts(3) w tz & Time when the job was last successfully matched with a resource.\\ \hline
    numjobmatches & integer & integer & Number of times the negotiator matches the job with a resource.\\ \hline
    jobstartdate & ts(3) w tz & ts(3) w tz & Time when the job first began running.\\ \hline
    jobcurrentstartdate & ts(3) w tz & ts(3) w tz & Time when the job's current run started.\\ \hline
    jobruncount & integer & integer & Number of times a shadow has been started for the job.\\ \hline
    filereadcount & numeric(38) & numeric(38) & Number of read(2) calls the job made (only standard universe).\\ \hline
    filereadbytes & numeric(38) & numeric(38) & Number of bytes read by the job (only standard universe).\\ \hline
    filewritecount & numeric(38) & numeric(38) & Number of write calls the job made (only standard universe).\\ \hline
    filewritebytes & numeric(38) & numeric(38) & Number of bytes written by the job (only standard universe).\\ \hline
    \multicolumn{4}{|c|}{\textbf{Table Continues on Next Page}}\\ \hline
  \end{tabular}
\vspace{24pt}

% jobs_horizontal_history Part 3
  \begin{tabular}{|l|l|l|p{2.6in}|}\hline
    \multicolumn{4}{|c|}{\textbf{Attributes of jobs\_horizontal\_history Table -- Part 3 of 3}}
\\ \hline
    \textbf{Name} & \textbf{O. Type} & \textbf{P. Type} & \textbf{Description}\\ \hline
    fileseekcount & numeric(38) & numeric(38) & Number of seek calls that this job made (only standard universe).\\ \hline
    totalsuspensions & integer & integer & Number of times the job has been suspended during its lifetime\\ \hline
    imagesize & numeric(38) & numeric(38) & Estimate of memory image size of the job in kilobytes.\\ \hline
    exitstatus & integer & integer & No longer used by Condor.\\ \hline
    localusercpu & numeric(38) & numeric(38) & Number of seconds of user CPU time the job used on the submit machine.\\ \hline
    localsyscpu & numeric(38) & numeric(38) & Number of seconds of system CPU time the job used on the submit machine.\\ \hline
    remoteusercpu & numeric(38) & numeric(38) & Number of seconds of user CPU time the job used on remote machines.\\ \hline
    remotesyscpu & numeric(38) & numeric(38) & Number of seconds of system CPU time the job used on remote machines.\\ \hline
    bytessent & numeric(38) & numeric(38) & Number of bytes sent to the job.\\ \hline
    bytesrecvd & numeric(38) & numeric(38) & Number of bytes received by the job.\\ \hline
    rscbytessent & numeric(38) & numeric(38) & Number of remote system call bytes sent to the job.\\ \hline
    rscbytesrecvd & numeric(38) & numeric(38) & Number of remote system call bytes received by the job.\\ \hline
    exitcode & integer & integer & Exit return code of the user job. Used when a job exits by means other than a signal.\\ \hline
    jobstatus & integer & integer & Current status of the job.\\ \hline
    enteredcurrentstatus & ts(3) w tz & ts(3) w tz & Time the job entered into its current status.\\ \hline
    remotewallclocktime & numeric(38) & numeric(38) & Cumulative number of seconds the job has been allocated to a machine.\\ \hline
    lastremotehost & varchar(4000) & varchar(4000) & The remote host for the last run of the job.\\ \hline
    completiondate & ts(3) w tz & ts(3) w tz & Time when the job completed; 0 if job has not yet completed.\\ \hline
    enteredhistorytable & ts(3) w tz & ts(3) w tz & Time when the job entered the history table.\\ \hline
    \multicolumn{4}{|l|}{\textbf{PRIMARY KEY:} (scheddname, scheddbirthdate, cluster\_id, proc\_id)}\\ \hline
    \multicolumn{4}{|l|}{\textbf{NOT NULL:} scheddname, scheddbirthdate, cluster\_id, and proc\_id cannot be null}\\ \hline
    \multicolumn{4}{|l|}{\textbf{INDEX:} Index named hist\_h\_i\_owner on owner}\\ \hline
  \end{tabular}
\vspace{24pt}

% jobs_vertical_history
  \begin{tabular}{|l|l|l|p{2.9in}|}\hline
    \multicolumn{4}{|c|}{\textbf{Attributes of jobs\_vertical\_history Table}}\\ \hline
    \textbf{Name} & \textbf{O. Type} & \textbf{P. Type} & \textbf{Description}\\ \hline
    scheddname & varchar(4000) & varchar(4000) & Name of the schedd that submitted the job.\\ \hline
    scheddbirthdate & integer & integer & The birth date of the schedd where the job is submitted.\\ \hline
    cluster\_id & integer & integer & Cluster identifier for the job.\\ \hline
    proc\_id & integer & integer & Process identifier for the job.\\ \hline
    attr & varchar(2000) & varchar(2000) & Attribute name.\\ \hline
    val & clob & text & Attribute value.\\ \hline
    \multicolumn{4}{|l|}{\textbf{PRIMARY KEY:} (scheddname, scheddbirthdate, cluster\_id, proc\_id, attr)}\\ \hline
    \multicolumn{4}{|l|}{\textbf{NOT NULL:} scheddname, scheddbirthdate, cluster\_id, proc\_id, and attr cannot be null}\\ \hline
  \end{tabular}
\vspace{24pt}

% procads_horizontal
  \begin{tabular}{|l|l|l|p{2.6in}|}\hline
    \multicolumn{4}{|c|}{\textbf{Attributes of procads\_horizontal Table}}\\ \hline
    \textbf{Name} & \textbf{O. Type} & \textbf{P. Type} & \textbf{Description}\\ \hline
    scheddname & varchar(4000) & varchar(4000) & Name of the schedd that submitted the job.\\ \hline
    cluster\_id & integer & integer & Cluster identifier for the job.\\ \hline
    proc\_id & integer & integer & Process identifier for the job.\\ \hline
    jobstatus & integer & integer & Current status of the job.\\ \hline
    imagesize & numeric(38) & numeric(38) & Estimate of memory image size of the job in kilobytes.\\ \hline
    remoteusercpu & numeric(38) & numeric(38) & Total number of seconds of user CPU time the job used on remote machines.\\ \hline
    remotewallclocktime & numeric(38) & numeric(38) & Cumulative number of seconds the job has been allocated to a machine.\\ \hline
    remotehost & varchar(4000) & varchar(4000) & Name of the machine running the job.\\ \hline
    globaljobid & varchar(4000) & varchar(4000) & Unique global identifier for the job.\\ \hline
    jobprio & integer & integer & Priority of the job.\\ \hline
    args & clob & text & Arguments passed to the job.\\ \hline
    shadowbday & ts(3) w tz & ts(3) w tz & The time when the shadow was started.\\ \hline
    enteredcurrentstatus & ts(3) w tz & ts(3) w tz & Time the job entered its current status.\\ \hline
    numrestarts & integer & integer & Number of times the job has restarted.\\ \hline
    \multicolumn{4}{|l|}{\textbf{PRIMARY KEY:} (scheddname, cluster\_id, proc\_id)} \\ \hline
    \multicolumn{4}{|l|}{\textbf{NOT NULL:} scheddname, cluster\_id, and proc\_id cannot be null} \\ \hline
  \end{tabular}
\vspace{24pt}

% procads_vertical
  \begin{tabular}{|l|l|l|p{3.2in}|}\hline
    \multicolumn{4}{|c|}{\textbf{Attributes of procads\_vertical Table}}\\ \hline
    \textbf{Name} & \textbf{O. Type} & \textbf{P. Type} & \textbf{Description}\\ \hline
    scheddname & varchar(4000) & varchar(4000) & Name of the schedd that submitted the job.\\ \hline
    cluster\_id & integer & integer & Cluster identifier for the job.\\ \hline
    proc\_id & integer & integer & Process identifier for the job.\\ \hline
    attr & varchar(2000) & varchar(2000) & Attribute name.\\ \hline
    val & clob & text & Attribute value.\\ \hline
  \end{tabular}
\vspace{24pt}
\end{center}


%%% MACHINES TABLES
\subsubsection{Machines Tables}
\begin{center}
% machines_horizontal Part 1 of 2
  \begin{tabular}{|l|l|l|p{2.4in}|}\hline
    \multicolumn{4}{|c|}{\textbf{Attributes of machines\_horizontal Table -- Part 1 of 2}}\\ \hline
    \textbf{Name} & \textbf{O. Type} & \textbf{P. Type} & \textbf{Description}\\ \hline
    machine\_id & varchar(4000) & varchar(4000) & Unique identifier of the machine.\\ \hline
    opsys & varchar(4000) & varchar(4000) & Operating system running on the machine.\\ \hline
    arch & varchar(4000) & varchar(4000) & Architecture of the machine.\\ \hline
    state & varchar(4000) & varchar(4000) & Condor state of the machine.\\ \hline
    activity & varchar(4000) & varchar(4000) & Condor job activity on the machine.\\ \hline
    keyboardidle & integer & integer & Number of seconds since activity has been detected on any keyboard or mouse associated with the machine.\\ \hline
    consoleidle & integer & integer & Number of seconds since activity has been detected on the console keyboard or mouse.\\ \hline
    loadavg & real & real & Current load average of the machine.\\ \hline
    condorloadavg & real & real & Portion of load average generated by Condor\\ \hline
    totalloadavg & real & real & \\ \hline
    virtualmemory & integer & integer & Amount of currently available virtual memory in kilobytes.\\ \hline
    memory & integer & integer & Amount of {RAM} in megabytes.\\ \hline
    totalvirtualmemory & integer & integer & \\ \hline
    cpubusytime & integer & integer & Time in seconds since cpuisbusy became true.\\ \hline
    cpuisbusy & varchar(5) & varchar(5) & (boolean) True when the CPU is busy.\\ \hline
    currentrank & real & real & The machine owner's affinity for running the Condor job which it is currently hosting.\\ \hline
    clockmin & integer & integer & Number of minutes passed since midnight.\\ \hline
    clockday & integer & integer & The day of the week.\\ \hline
    lastreportedtime & ts(3) w tz & ts(3) w tz & Time when the Condor central manager last received a status update from this machine.\\ \hline
    enteredcurrentactivity & ts(3) w tz & ts(3) w tz & Time when the machine entered the current activity.\\ \hline
    enteredcurrentstate & ts(3) w tz & ts(3) w tz & Time when the machine entered the current state.\\ \hline
    updatesequencenumber & integer & integer & Each update includes a sequence number.\\ \hline
    \multicolumn{4}{|c|}{\textbf{Table Continues on Next Page}}\\ \hline
  \end{tabular}
\vspace{24pt}

% machines_horizontal Part 2 of 2
  \begin{tabular}{|l|l|l|p{2.4in}|}\hline
    \multicolumn{4}{|c|}{\textbf{Attributes of machines\_horizontal Table -- Part 2 of 2}}\\ \hline
    updatestotal & integer & integer & The number of updates received from the daemon.\\ \hline
    updatessequenced & integer & integer & The number of updates that were in order.\\ \hline
    updateslost & integer & integer & The number of updates that were lost.\\ \hline
    globaljobid & varchar(4000) & varchar(4000) & Unique global identifier for the job.\\ \hline
    lastreportedtime\_epoch & integer & integer & The equivalent epoch time of lastreportedtime.\\ \hline
    \multicolumn{4}{|l|}{\textbf{PRIMARY KEY:} machine\_id} \\ \hline    
  \end{tabular}
\vspace{24pt}

% machines_horizontal_history Part 1 of 2
  \begin{tabular}{|l|l|l|p{2.4in}|}\hline
    \multicolumn{4}{|c|}{\textbf{Attributes of machines\_horizontal\_history Table -- Part 1 of 2}}\\ \hline
    \textbf{Name} & \textbf{O. Type} & \textbf{P. Type} & \textbf{Description}\\ \hline
    machine\_id & varchar(4000) & varchar(4000) & Unique identifier of the machine.\\ \hline
    opsys & varchar(4000) & varchar(4000) & Operating system running on the machine.\\ \hline
    arch & varchar(4000) & varchar(4000) & Architecture of the machine.\\ \hline
    state & varchar(4000) & varchar(4000) & Condor state of the machine.\\ \hline
    activity & varchar(4000) & varchar(4000) & Condor job activity on the machine.\\ \hline
    keyboardidle & integer & integer & Number of seconds since activity has been detected on any keyboard or mouse associated with the machine.\\ \hline
    consoleidle & integer & integer & Number of seconds since activity has been detected on the console keyboard or mouse.\\ \hline
    loadavg & real & real & Current load average of the machine.\\ \hline
    condorloadavg & real & real & Portion of load average generated by Condor\\ \hline
    totalloadavg & real & real & \\ \hline
    virtualmemory & integer & integer & Amount of currently available virtual memory in kilobytes.\\ \hline
    memory & integer & integer & Amount of {RAM} in megabytes.\\ \hline
    totalvirtualmemory & integer & integer & \\ \hline
    cpubusytime & integer & integer & Time in seconds since cpuisbusy became true.\\ \hline
    cpuisbusy & varchar(5) & varchar(5) & (boolean) True when the CPU is busy.\\ \hline
    currentrank & real & real & The machine owner's affinity for running the Condor job which it is currently hosting.\\ \hline
    clockmin & integer & integer & Number of minutes passed since midnight.\\ \hline
    clockday & integer & integer & The day of the week.\\ \hline
    lastreportedtime & ts(3) w tz & ts(3) w tz & Time when the Condor central manager last received a status update from this machine.\\ \hline
    enteredcurrentactivity & ts(3) w tz & ts(3) w tz & Time when the machine entered the current activity.\\ \hline
    enteredcurrentstate & ts(3) w tz & ts(3) w tz & Time when the machine entered the current state.\\ \hline
    updatesequencenumber & integer & integer & Each update includes a sequence number.\\ \hline
    \multicolumn{4}{|c|}{\textbf{Table Continues on Next Page}}\\ \hline
  \end{tabular}
\vspace{24pt}

% machines_horizontal_history Part 2 of 2
  \begin{tabular}{|l|l|l|p{2.7in}|}\hline
    \multicolumn{4}{|c|}{\textbf{Attributes of machines\_horizontal\_history Table -- Part 2 of 2}}\\ \hline
    \textbf{Name} & \textbf{O. Type} & \textbf{P. Type} & \textbf{Description}\\ \hline
    updatestotal & integer & integer & The number of updates received from the daemon.\\ \hline
    updatessequenced & integer & integer & The number of updates that were in order.\\ \hline
    updateslost & integer & integer & The number of updates that were lost.\\ \hline
    globaljobid & varchar(4000) & varchar(4000) & Unique global identifier for the job.\\ \hline
    end\_time & ts(3) w tz & ts(3) w tz & The end of when the \ca\ is valid.\\ \hline
  \end{tabular}
\vspace{24pt}


% machines_vertical
  \begin{tabular}{|l|l|l|p{3.2in}|}\hline
    \multicolumn{4}{|c|}{\textbf{Attributes of machines\_vertical Table}}\\ \hline
    \textbf{Name} & \textbf{O. Type} & \textbf{P. Type} & \textbf{Description}\\ \hline
    machine\_id & varchar(4000) & varchar(4000) & Unique identifier of the machine.\\ \hline
    attr & varchar(2000) & varchar(2000) & Attribute name.\\ \hline
    val & clob & text & Attribute value.\\ \hline
    start\_time & ts(3) w tz & ts(3) w tz & Time when this attribute--value pair became valid.\\ \hline
    \multicolumn{4}{|l|}{\textbf{PRIMARY KEY:} (machine\_id, attr)} \\ \hline
    \multicolumn{4}{|l|}{\textbf{NOT NULL:} machine\_id and attr cannot be null} \\ \hline
  \end{tabular}
\vspace{24pt}

% machines_vertical_history
  \begin{tabular}{|l|l|l|p{3.2in}|}\hline
    \multicolumn{4}{|c|}{\textbf{Attributes of machines\_vertical\_history Table}}\\ \hline
    \textbf{Name} & \textbf{O. Type} & \textbf{P. Type} & \textbf{Description}\\ \hline
    machine\_id & varchar(4000) & varchar(4000) & Unique identifier of the machine.\\ \hline
    attr & varchar(4000) & varchar(4000) & Attribute name.\\ \hline
    val & clob & text & Attribute value.\\ \hline
    start\_time & ts(3) w tz & ts(3) w tz & Time when this attribute--value pair became valid.\\ \hline
    end\_time & ts(3) w tz & ts(3) w tz & Time when this attribute--value pair became invalid.\\ \hline
  \end{tabular}
\vspace{24pt}
\end{center}


%%% MATCHMAKING TABLES
\subsubsection{Matchmaking Tables}
\begin{center}
% matches
  \begin{tabular}{|l|l|l|p{3.0in}|}\hline
    \multicolumn{4}{|c|}{\textbf{Attributes of matches Table}}\\ \hline
    \textbf{Name} & \textbf{O. Type} & \textbf{P. Type} & \textbf{Description}\\ \hline
    match\_time & ts(3) w tz & ts(3) w tz & Time the match was made.\\ \hline
    username & varchar(4000) & varchar(4000) & User who submitted the job.\\ \hline
    scheddname & varchar(4000) & varchar(4000) & Name of the schedd that the job is submitted to.\\ \hline
    cluster\_id & integer & integer & Cluster identifier for the job.\\ \hline
    proc\_id & integer & integer & Process identifier for the job.\\ \hline
    globaljobid & varchar(4000) & varchar(4000) & Unique global identifier for the job.\\ \hline
    machine\_id & varchar(4000) & varchar(4000) & Identifier of the machine the job matched with.\\ \hline
    remote\_user & varchar(4000) & varchar(4000) & User that was preempted.\\ \hline
    remote\_priority & real & real & The preempted user's priority.\\ \hline
  \end{tabular}
\vspace{24pt}

% rejects
  \begin{tabular}{|l|l|l|p{3.2in}|}\hline
    \multicolumn{4}{|c|}{\textbf{Attributes of rejects Table}}\\ \hline
    \textbf{Name} & \textbf{O. Type} & \textbf{P. Type} & \textbf{Description}\\ \hline
    reject\_time & ts(3) w tz & ts(3) w tz & Time when the job was rejected.\\ \hline
    username & varchar(4000) & varchar(4000) & User who submitted the job.\\ \hline
    scheddname & varchar(4000) & varchar(4000) & Name of the schedd that submitted the job.\\ \hline
    cluster\_id & integer & integer & Cluster identifier for the job.\\ \hline
    proc\_id & integer & integer & Process identifier for the job.\\ \hline    
    globaljobid & varchar(4000) & varchar(4000) & Unique global identifier for the job.\\ \hline
  \end{tabular}
\vspace{24pt}
\end{center}


%%% RUNTIME TABLES
\subsubsection{Runtime Tables}
\begin{center}
% events
  \begin{tabular}{|l|l|l|p{3.2in}|}\hline
    \multicolumn{4}{|c|}{\textbf{Attributes of events Table}}\\ \hline
    \textbf{Name} & \textbf{O. Type} & \textbf{P. Type} & \textbf{Description}\\ \hline
    scheddname & varchar(4000) & varchar(4000) & Name of the schedd that submitted the job.\\ \hline
    cluster\_id & integer & integer & Cluster identifier for the job.\\ \hline
    proc\_id & integer & integer & Process identifier for the job.\\ \hline
    globaljobid & varchar(4000) & varchar(4000) & Global identifier of the job that generated the event.\\ \hline
    run\_id & numeric(12,0) & numeric(12,0) & Identifier of the run that the event is associated with.\\ \hline
    eventtype & integer & integer & Numeric type code of the event.\\ \hline
    eventtime & ts(3) w tz & ts(3) w tz & Time the event occurred.\\ \hline
    description & varchar(4000) & varchar(4000) & Description of the event.\\ \hline
  \end{tabular}
\vspace{24pt}

% generic_messages
  \begin{tabular}{|l|l|l|p{3.3in}|}\hline
    \multicolumn{4}{|c|}{\textbf{Attributes of generic\_messages Table}}\\ \hline
    \textbf{Name} & \textbf{O. Type} & \textbf{P. Type} & \textbf{Description}\\ \hline
    eventtype & varchar(4000) & varchar(4000) & The type of event.\\ \hline
    eventkey & varchar(4000) & varchar(4000) & The key of the event.\\ \hline
    eventtime & ts(3) w tz & ts(3) w tz & The time of the event.\\ \hline
    eventloc & varchar(4000) & varchar(4000) & The location of the event.\\ \hline
    attname & varchar(4000) & varchar(4000) & The attribute name.\\ \hline
    attval & clob & text & The attribute value.\\ \hline
    attrtype & varchar(4000) & varchar(4000) & The attribute type.\\ \hline
  \end{tabular}
\vspace{24pt}

% runs
  \begin{tabular}{|l|l|l|p{2.5in}|}\hline
    \multicolumn{4}{|c|}{\textbf{Attributes of runs Table}}\\ \hline
    \textbf{Name} & \textbf{O. Type} & \textbf{P. Type} & \textbf{Description}\\ \hline
    run\_id & numeric(12) & numeric(12) & Unique identifier of the run.\\ \hline
    machine\_id & varchar(4000) & varchar(4000) & Identifier of the machine where the job ran.\\ \hline
    scheddname & varchar(4000) & varchar(4000) & Name of the schedd that submitted the job.\\ \hline
    cluster\_id & integer & integer & Cluster identifier for the job.\\ \hline
    proc\_id & integer & integer & Process identifier for the job.\\ \hline
    spid & integer & integer & Subprocess identifier for the job. \\ \hline
    globaljobid & varchar(4000) & varchar(4000) & Identifier of the job that was run.\\ \hline 
    startts & ts(3) w tz & ts(3) w tz & Time when the job started.\\ \hline
    endts & ts(3) w tz & ts(3) w tz & Time when the job ended.\\ \hline
    endtype & smallint & smallint & The type of ending event.\\ \hline
    endmessage & varchar(4000) & varchar(4000) & The ending message.\\ \hline
    wascheckpointed & varchar(7) & varchar(7) & Whether the run was checkpointed.\\ \hline
    imagesize & numeric(38) & numeric(38) & The image size of the executable.\\ \hline
    runlocalusageuser & integer & integer & The time the job spent in usermode on execute machines (only standard universe).\\ \hline
    runlocalusagesystem & integer & integer & The time the job was in system calls.\\ \hline
    runremoteusageuser & integer & integer & The time the shadow spent working for the job.\\ \hline
    runremoteusagesystem & integer & integer & The time the shadow spent in system calls for the job.\\ \hline
    runbytessent & numeric(38) & numeric(38) & Number of bytes sent to the run.\\ \hline
    runbytesreceived & numeric(38) & numeric(38) & Number of bytes received from the run.\\ \hline
    \multicolumn{4}{|l|}{\textbf{PRIMARY KEY:} run\_id} \\ \hline    
    \multicolumn{4}{|l|}{\textbf{NOT NULL:} run\_id cannot be null} \\ \hline    
  \end{tabular}
\vspace{24pt}
\end{center}


%%% SYSTEM TABLES
\subsubsection{System Tables}
\begin{center}
% dummy_single_row_table
  \begin{tabular}{|l|l|l|p{4in}|}\hline
    \multicolumn{4}{|c|}{\textbf{Attributes of dummy\_single\_row\_table Table}}\\ \hline
    \textbf{Name} & \textbf{O. Type} & \textbf{P. Type} & \textbf{Description}\\ \hline
    a & varchar(1) & varchar(1) & A dummy column. \\ \hline
  \end{tabular}
\vspace{24pt}

% history_jobs_to_purge
  \begin{tabular}{|l|l|l|p{3.2in}|}\hline
    \multicolumn{4}{|c|}{\textbf{Attributes of history\_jobs\_to\_purge Table}}\\ \hline
    scheddname & varchar(4000) & varchar(4000) & Name of the schedd that submitted the job. \\ \hline
    cluster\_id & integer & integer & Cluster identifier for the job. \\ \hline
    proc\_id & integer & integer & Process identifier for the job. \\ \hline
    globaljobid & varchar(4000) & varchar(4000) & Unique global identifier for the job. \\ \hline
  \end{tabular}
\vspace{24pt}

% jobqueuepollinginfo
  \begin{tabular}{|l|l|l|p{2.6in}|}\hline
    \multicolumn{4}{|c|}{\textbf{Attributes of jobqueuepollinginfo Table}}\\ \hline
    \textbf{Name} & \textbf{O. Type} & \textbf{P. Type} & \textbf{Description}\\ \hline
    scheddname & varchar(4000) & varchar(4000) & Name of the schedd that submitted the job.\\ \hline
    last\_file\_mtime & integer & integer & The last modification time of the file.\\ \hline
    last\_file\_size & numeric(38) & numeric(38) & The last size of the file in bytes.\\ \hline
    last\_next\_cmd\_offset & integer & integer & The last offset for the next command.\\ \hline
    last\_cmd\_offset & integer & integer & The last offset of the current command.\\ \hline
    last\_cmd\_type & smallint & smallint & The last type of command.\\ \hline
    last\_cmd\_key & varchar(4000) & varchar(4000) & The last key of the command.\\ \hline
    last\_cmd\_mytype & varchar(4000) & varchar(4000) & The last my \ca\ type of the command.\\ \hline
    last\_cmd\_targettype & varchar(4000) & varchar(4000) & The last target \ca\ type.\\ \hline
    last\_cmd\_name & varchar(4000) & varchar(4000) & The attribute name of the command.\\ \hline
    last\_cmd\_value & varchar(4000) & varchar(4000) & The attribute value of the command.\\ \hline
  \end{tabular}
\end{center}

%\end{document}

