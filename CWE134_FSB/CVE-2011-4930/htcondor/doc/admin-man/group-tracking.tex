%%%%%%%%%%%%%%%%%%%%%%%%%%%%%%%%%%%%%%%%%%%%%%%%%%%%%%%%%%%%%%%%%%%%%%%%%%%
\subsection{\label{sec:GroupTracking}Group ID-Based Process Tracking} 
%%%%%%%%%%%%%%%%%%%%%%%%%%%%%%%%%%%%%%%%%%%%%%%%%%%%%%%%%%%%%%%%%%%%%%%%%%%

One function that Condor often must perform is keeping track of all
processes created by a job. This is done so that Condor can provide
resource usage statistics about jobs, and also so that Condor can properly
clean up any processes that jobs leave behind when they exit.

In general, tracking process families is difficult to do reliably.
By default Condor uses a combination of process parent-child
relationships, process groups, and information that Condor places in a
job's environment to track process families on a best-effort
basis. This usually works well, but it can falter for certain
applications or for jobs that try to evade detection.

Jobs that run with a user account dedicated for Condor's use
can be reliably tracked, since all Condor needs to do is look for all
processes running using the given account. Administrators must specify
in Condor's configuration what accounts can be considered dedicated
via the \Macro{DEDICATED\_EXECUTE\_ACCOUNT\_REGEXP} setting. See
Section~\ref{sec:RunAsNobody} for further details.

Ideally, jobs can be reliably tracked regardless of the user account
they execute under. This can be accomplished with group ID-based
tracking. This method of tracking requires that a range of dedicated
\emph{group} IDs (GID) be set aside for Condor's use. The number of GIDs
that must be set aside for an execute machine is equal to its number
of execution slots. GID-based tracking is only available on Linux, and
it requires that Condor either runs as \Login{root} or uses privilege
separation (see Section~\ref{sec:PrivSep}).

GID-based tracking works by placing a dedicated GID in the
supplementary group list of a job's initial process. Since modifying
the supplementary group ID list requires
\Login{root} privilege, the job will not be able to create processes
that go unnoticed by Condor.

Once a suitable GID range has been set aside for process tracking,
GID-based tracking can be enabled via the
\Macro{USE\_GID\_PROCESS\_TRACKING} parameter. The minimum and maximum
GIDs included in the range are specified with the
\Macro{MIN\_TRACKING\_GID} and \Macro{MAX\_TRACKING\_GID}
settings. For example, the following would enable GID-based tracking
for an execute machine with 8 slots.
\begin{verbatim}
USE_GID_PROCESS_TRACKING = True
MIN_TRACKING_GID = 750
MAX_TRACKING_GID = 757
\end{verbatim}

If the defined range is too small, such that there is not a GID available
when starting a job,
then the \Condor{starter} will fail as it tries to start the job.
An error message will be logged stating that there are no more tracking GIDs.

GID-based process tracking requires use of the \Condor{procd}. If
\MacroNI{USE\_GID\_PROCESS\_TRACKING} is true, the \Condor{procd} will
be used regardless of the \Macro{USE\_PROCD} setting.  Changes to
\MacroNI{MIN\_TRACKING\_GID} and \MacroNI{MAX\_TRACKING\_GID} require
a full restart of Condor.
