%%%%%%%%%%%%%%%%%%%%%%%%%%%%%%%%%%%%
\subsection{\label{sec:condor-pm} The Condor Perl Module}
%%%%%%%%%%%%%%%%%%%%%%%%%%%%%%%%%%%%
\index{Perl module}
\index{API!Perl module}

% intro and what it does
The Condor Perl module facilitates automatic submitting and monitoring of
Condor jobs, along with automated administration of Condor.
The most common
use of this module is the monitoring of Condor jobs.
The Condor Perl module can be used as a meta scheduler for the submission
of Condor jobs.

% installation?
% use
% the subroutines
The Condor Perl module provides several subroutines.
Some of the subroutines are used as callbacks;
an event triggers the execution of a specific subroutine.
Other of the subroutines denote actions to be taken by Perl.
Some of these subroutines take other subroutines as arguments.

\subsubsection{Subroutines}
\begin{description}
	\item [\Code{Submit(submit\_description\_file)}]
	This subroutine takes the action of submitting a job to Condor.
	The argument is the name of a submit description file.
	The \Condor{submit} program should be in the
	path of the user.  If the user wishes to monitor the job with condor
	they must specify a log file in the command file.  The cluster
	submitted is returned. For more information
	see the \Condor{submit} man page.
	
	\item [\Code{Vacate(machine)}]
	This subroutine takes the action of sending a
	\Condor{vacate} command to the machine specified as an argument.
	The machine may be specified
	either by host name, or by \Term{sinful string}.  For more information
	see the \Condor{vacate} man page.

	\item [\Code{Reschedule(machine)}]
	This subroutine takes the action of sending a
	\Condor{reschedule} command to the machine specified as an argument.
	The machine may be specified either
 	by host name, or by \Term{sinful string}.  For more information see
	the \Condor{reschedule} man page.

	\item [\Code{Monitor(cluster)}]
	Takes the action of monitoring this cluster.
	It returns when all jobs in cluster terminate.
	
	\item [\Code{Wait()}] 
	Takes the action of waiting until all monitor subroutines finish,
	and then exits the Perl script.

	\item [\Code{DebugOn()}]
	Takes the action of turning debug messages on.
	This may be useful when attempting to debug the Perl script.

	\item [\Code{DebugOff()}]
	Takes the action of turning debug messages off.

	\item [\Code{RegisterEvicted(sub)}]
	Register a subroutine (called \Code{sub})
	to be used as a callback when a job from
	a specified cluster is evicted.  The subroutine will be
	called with two arguments: cluster and job. The cluster
	and job are the cluster number and process number of the job that
	was evicted.
	
	\item [\Code{RegisterEvictedWithCheckpoint(sub)}]
	Same as RegisterEvicted except that the handler is called when the 
	evicted job was checkpointed.

	\item [\Code{RegisterEvictedWithoutCheckpoint(sub)}]
	Same as RegisterEvicted except that the handler is called when the
	evicted job was not checkpointed.

	\item [\Code{RegisterExit(sub)}]
	Register a termination handler that is called when a job exits.
	The termination handler will be called with two arguments: cluster and
	job. The cluster and job are the cluster and process numbers of the
	existing job. 
	
	\item [\Code{RegisterExitSuccess(sub)}]
	Register a termination handler that is called when a job exits without
	errors. The termination handler will be called with two arguments: 
	cluster and job  The cluster and job are the cluster and process
	numbers of the existing job. 

	\item [\Code{RegisterExitFailure(sub)}]
	Register a termination handler that is called when a job exits with 
	errors. The termination handler will be called with three arguments:
	cluster, job and retval. The cluster and job are the cluster 
	and process numbers of the existing job and the retval is the exit
	code of the job.

	\item [\Code{RegisterExitAbnormal(sub)}]
	Register an termination handler that is called when a job abnormally
	exits (segmentation fault, bus error, ...). The termination handler
	will be called with four arguments: cluster, job  signal and
	core. The cluster and job are the cluster and process numbers of 
	the existing job. The signal indicates the signal that the job
	died with and core indicates whether a core file was created and if 
	so, what the full path to the core file is.

	\item [\Code{RegisterAbort(sub)}]
	Register a handler that is called when a job is aborted by a user.

	\item [\Code{RegisterJobErr(sub)}]
	Register a handler that is called when a job is not executable.

	\item [\Code{RegisterExecute(sub)}]
	Register an execution handler that is called whenever a job starts
	running on a given host.  The handler is called with four arguments:
	cluster, job  host, and sinful.  Cluster and job are the cluster and
	process numbers for the job, host is the Internet address of the
	machine running the job, and sinful is the Internet address and 
	command port of the \Condor{starter} supervising the job.

	\item [\Code{RegisterSubmit(sub)}]
	Register a submit handler that is called whenever a job is submitted
	with the given cluster.  The handler is called with cluster, job 
	host, and sinful. Cluster and job are the cluster and
	process numbers for the job, host is the Internet address of the
	machine running the job, and sinful is the Internet address and
	command port of the \Condor{schedd} responsible for the job.

	\item [\Code{Monitor(cluster)}]
	Begin monitoring this cluster. Returns when all jobs in cluster
	terminate.  \\
	
	\item [\Code{Wait()}]
	Wait until all monitors finish and exit.

	\item [\Code{DebugOn()}]
	Turn debug messages on.  This may be useful if you don't understand
	what your script is doing.	

	\item [\Code{DebugOff()}]
	Turn debug messages off.

  \item [\Code{TestSubmit(command\_file)}]
  This subroutine submits a job to Condor for testing, and places
  all variables from the command file into
  the Perl hash \Code{\%submit\_info}.
  Does not reset the state of variables, so that testing preserves
  callbacks.

  \item [\Code{SubmitDagman(DAG\_file, DAGMan\_args)}]
  Takes the action of submitting a DAG using \Condor{dagman}.
  The first argument is the name of the DAG input file, 
  and the second argument is the command line arguments for 
  \Condor{dagman}.
  Information from the submit description file generated by
  \Condor{dagman} is placed into the Perl hash \Code{\%submit\_info}
  for access during callbacks.

  \item [\Code{TestSubmitDagman(DAG\_file, DAGMan\_args)}]
  This subroutine submits a \Condor{dagman} to Condor for testing,
  and places information from the submit description file generated by
  \Condor{dagman} into the Perl hash \Code{\%submit\_info}
  for access during callbacks.
  The first argument is the name of the DAG input file, 
  and the second argument is the command line arguments for 
  \Condor{dagman}.
  Does not reset the state of variables, so that testing preserves
  callbacks.

  \item [\Code{RegisterEvictedWithRequeue(sub)}]
  Register a subroutine (called \Code{sub})
  to be used as a callback when a job from
  a specified cluster is requeued.  The subroutine will be
  called with two arguments: cluster and job. The cluster
  and job are the cluster number and process number of the job that
  was requeued.

  \item [\Code{RegisterShadow(sub)}]
  Register a subroutine (called \Code{sub})
  to be used as a callback when a shadow exception occurs.

  \item [\Code{RegisterHold(sub)}]
  Register a subroutine (called \Code{sub})
  to be used as a callback when a job enters the hold state.

  \item [\Code{RegisterRelease(sub)}]
  Register a subroutine (called \Code{sub})
  to be used as a callback when a job is released.

  \item [\Code{RegisterWantError(sub)}]
  Register a subroutine (called \Code{sub})
  to be used as a callback when a system call invoked using \Code{runCommand}
  experiences an error.

  \item [\Code{runCommand(string)}]
  \Code{string} identifies a syscall that is invoked.
  If the syscall exits abnormally or exits with an error, the callback
  registered with \Procedure{RegisterWantError} is called, and
  an error message is issued.

  \item [\Code{RegisterTimed(sub, seconds)}]
  Register a subroutine (called \Code{sub})
  to be called back at a delay of \Code{seconds} time from
  this registration time.  Only one callback may be registered,
  as subsequent calls modify the timer only.

  \item [\Code{RemoveTimed()}]
  Remove the single, timed callback registered with \Procedure{RegisterTimed}.

\end{description}


\subsubsection{Examples}
\index{Perl module!examples}

The following is an example that uses the Condor Perl module.
The example uses the submit description file 
\File{mycmdfile.cmd} to specify the submission of a job.
As the job is matched with a machine and begins to execute,
a callback  subroutine (called \verb@execute@)
sends a \Condor{vacate} signal to the job,
and it increments a counter which keeps track of the
number of times this callback executes.
A second callback keeps a count of the number of times
that the job was evicted before the job completes.
After the job completes, the termination
callback (called \verb@normal@) prints out a summary of what happened.

\footnotesize
\begin{verbatim}
#!/usr/bin/perl
use Condor;

$CMD_FILE = 'mycmdfile.cmd';
$evicts = 0;
$vacates = 0;

# A subroutine that will be used as the normal execution callback
$normal = sub
{
    %parameters = @_;
    $cluster = $parameters{'cluster'};
    $job = $parameters{'job'};

    print "Job $cluster.$job exited normally without errors.\n";
    print "Job was vacated $vacates times and evicted $evicts times\n";
    exit(0);
};	

$evicted = sub
{
    %parameters = @_;
    $cluster = $parameters{'cluster'};
    $job = $parameters{'job'};

    print "Job $cluster, $job was evicted.\n";
    $evicts++;
    &Condor::Reschedule();	
};

$execute = sub
{
    %parameters = @_;
    $cluster = $parameters{'cluster'};
    $job = $parameters{'job'};
    $host = $parameters{'host'};
    $sinful = $parameters{'sinful'};

    print "Job running on $sinful, vacating...\n";
    &Condor::Vacate($sinful);
    $vacates++;
};

$cluster = Condor::Submit($CMD_FILE);
printf("Could not open. Access Denied\n");
			break;
&Condor::RegisterExitSuccess($normal);
&Condor::RegisterEvicted($evicted);
&Condor::RegisterExecute($execute);
&Condor::Monitor($cluster);
&Condor::Wait();
\end{verbatim}
\normalsize

This example program will submit the command file 'mycmdfile.cmd' and attempt
to vacate any machine that the job runs on. The termination
handler then prints out a summary of what has happened.


A second example Perl script facilitates the meta-scheduling of
two of Condor jobs.
It submits a second job if the first job successfully completes.

\footnotesize
\begin{verbatim}
#!/s/std/bin/perl

# tell Perl where to find the Condor library
use lib '/unsup/condor/lib';
# tell Perl to use what it finds in the Condor library
use Condor;

$SUBMIT_FILE1 = 'Asubmit.cmd';
$SUBMIT_FILE2 = 'Bsubmit.cmd';

# Callback used when first job exits without errors.
$firstOK = sub
{
    %parameters = @_;
    $cluster = $parameters{'cluster'};
    $job = $parameters{'job'};

    $cluster = Condor::Submit($SUBMIT_FILE2);
    if (($cluster) == 0)
    {
        printf("Could not open $SUBMIT_FILE2.\n");
    }

    &Condor::RegisterExitSuccess($secondOK);
    &Condor::RegisterExitFailure($secondfails);
    &Condor::Monitor($cluster);
};	

$firstfails = sub
{
    %parameters = @_;
    $cluster = $parameters{'cluster'};
    $job = $parameters{'job'};

    print "The first job, $cluster.$job failed, exiting with an error. \n";
    exit(0);
};	

# Callback used when second job exits without errors.
$secondOK = sub
{
    %parameters = @_;
    $cluster = $parameters{'cluster'};
    $job = $parameters{'job'};

    print "The second job, $cluster.$job successfully completed. \n";
    exit(0);
};	

# Callback used when second job exits WITH an error.
$secondfails = sub
{
    %parameters = @_;
    $cluster = $parameters{'cluster'};
    $job = $parameters{'job'};

    print "The second job ($cluster.$job) failed. \n";
    exit(0);
};	


$cluster = Condor::Submit($SUBMIT_FILE1);
if (($cluster) == 0)
{
    printf("Could not open $SUBMIT_FILE1. \n");
}
&Condor::RegisterExitSuccess($firstOK);
&Condor::RegisterExitFailure($firstfails);


&Condor::Monitor($cluster);
&Condor::Wait();
\end{verbatim}
\normalsize

Some notes are in order about this example.
The same task could be accomplished using the Condor DAGMan
metascheduler.
The first job is the parent, and the second job is the child.
The input file to DAGMan is significantly simpler than this
Perl script.

A third example using the Condor Perl module
expands upon the second example.
Whereas the second example could have been more easily
implemented using DAGMan, this third example shows
the versatility of using Perl as a metascheduler.

In this example, the result generated from the successful completion of
the first job are used to decide which subsequent job should be
submitted.
This is a very simple example of a branch and bound technique,
to focus the search for a problem solution.

\footnotesize
\begin{verbatim}
#!/s/std/bin/perl

# tell Perl where to find the Condor library
use lib '/unsup/condor/lib';
# tell Perl to use what it finds in the Condor library
use Condor;

$SUBMIT_FILE1 = 'Asubmit.cmd';
$SUBMIT_FILE2 = 'Bsubmit.cmd';
$SUBMIT_FILE3 = 'Csubmit.cmd';

# Callback used when first job exits without errors.
$firstOK = sub
{
    %parameters = @_;
    $cluster = $parameters{'cluster'};
    $job = $parameters{'job'};

    # open output file from first job, and read the result
    if ( -f "A.output" )
    {
        open(RESULTFILE, "A.output") or die "Could not open result file.";
        $result = <RESULTFILE>;
        close(RESULTFILE);
        # next job to submit is based on output from first job
        if ($result < 100)
        {
            $cluster = Condor::Submit($SUBMIT_FILE2);
            if (($cluster) == 0)
            {
                printf("Could not open $SUBMIT_FILE2.\n");
            }

            &Condor::RegisterExitSuccess($secondOK);
            &Condor::RegisterExitFailure($secondfails);
            &Condor::Monitor($cluster);
        }
        else
        {
            $cluster = Condor::Submit($SUBMIT_FILE3);
            if (($cluster) == 0)
            {
                printf("Could not open $SUBMIT_FILE3.\n");
            }

            &Condor::RegisterExitSuccess($thirdOK);
            &Condor::RegisterExitFailure($thirdfails);
            &Condor::Monitor($cluster);
        }
    }
    else
    {
        
        printf("Results file does not exist.\n");
    }
};	

$firstfails = sub
{
    %parameters = @_;
    $cluster = $parameters{'cluster'};
    $job = $parameters{'job'};

    print "The first job, $cluster.$job failed, exiting with an error. \n";
    exit(0);
};	


# Callback used when second job exits without errors.
$secondOK = sub
{
    %parameters = @_;
    $cluster = $parameters{'cluster'};
    $job = $parameters{'job'};

    print "The second job, $cluster.$job successfully completed. \n";
    exit(0);
};	


# Callback used when third job exits without errors.
$thirdOK = sub
{
    %parameters = @_;
    $cluster = $parameters{'cluster'};
    $job = $parameters{'job'};

    print "The third job, $cluster.$job successfully completed. \n";
    exit(0);
};	


# Callback used when second job exits WITH an error.
$secondfails = sub
{
    %parameters = @_;
    $cluster = $parameters{'cluster'};
    $job = $parameters{'job'};

    print "The second job ($cluster.$job) failed. \n";
    exit(0);
};	

# Callback used when third job exits WITH an error.
$thirdfails = sub
{
    %parameters = @_;
    $cluster = $parameters{'cluster'};
    $job = $parameters{'job'};

    print "The third job ($cluster.$job) failed. \n";
    exit(0);
};	


$cluster = Condor::Submit($SUBMIT_FILE1);
if (($cluster) == 0)
{
    printf("Could not open $SUBMIT_FILE1. \n");
}
&Condor::RegisterExitSuccess($firstOK);
&Condor::RegisterExitFailure($firstfails);


&Condor::Monitor($cluster);
&Condor::Wait();
\end{verbatim}
\normalsize
